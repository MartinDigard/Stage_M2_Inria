\subsection*{Sujet du stage}
L'objectif de la transcription automatique de la musique (AMT) [1] est de convertir la performance d'un musicien en notation musicale - un peu comme la conversion de la parole en texte dans le traitement du langage naturel. Elle est considérée comme l'un des problèmes de recherche les plus anciens et les plus difficiles dans le domaine de la recherche d'information musicale (MIR).\\
Le cas de la transcription de la batterie (DT) est très particulier puisqu'il s'agit d'instruments sans hauteur, d'événements avec (presque) aucune durée et de notations spécifiques. Il a été la source de nombreuses études MIR, voir [2] pour un aperçu. La plupart de ces travaux se concentrent sur des méthodes de calcul pour la détection d'événements sonores de batterie à partir de signaux acoustiques, et sur l'extraction de caractéristiques de bas niveau telles que la classe d'instrument et le moment de l'apparition du son (peak picking). Cependant, très peu d'entre eux ont abordé la tâche de générer une notation musicale (rythmique) lisible à partir des caractéristiques ci-dessus, une étape cruciale dans un contexte musical et loin d'être triviale.\\\\
La présente proposition s'intéresse à ce dernier problème, et plus précisément à :

1. l'étude de modèles de langage (LM) incorporant certaines informations musicales de haut niveau nécessaires à la génération de partitions de qualité. On devrait en particulier considérer des hiérarchies d'événements de batterie induisant des placements temporels cohérents et se prêtant à des notations rythmiques faciles à lire pour un batteur entraîné ; voir [3] pour des modèles structurés en arbre basés sur la théorie formelle du langage, que nous développons dans le contexte d'outils AMT plus généraux.\\\\
2. l'intégration de ces LM avec l'état de l'art des modèles acoustiques (AM) et des méthodes pour les tâches de traitement du signal ci-dessus. Cela nécessite la prise en compte du contexte musical et des informations musicales de haut niveau des LM en plus des caractéristiques acoustiques de bas niveau ci-dessus.\\
En outre, certaines expériences seront menées sur la base d'ensembles de données publiques, afin d'évaluer l'approche intégrée. Elles devraient couvrir certains cas de motifs rythmiques complexes se chevauchant.\\

Au-delà de l'intégration de modèles, il sera également intéressant d'étudier comment l'utilisation de LM peut améliorer les résultats de l'AM, voir [2], et ouvrir la voie à la génération entièrement automatisée de partitions de batterie et au problème général de l'AMT de bout en bout.

\subsection*{Problématique traitée}	
	En entrée : midi (séquence d’événements datés (piano roll) accompagné d’une grammaire pondérée)\\
	$\Rightarrow$ parsing\\
	$\Rightarrow$ global parsing tree\\
	$\Rightarrow$ RI (Représentation Intermédiaire) arbres locaux par intruments\\
	$\Rightarrow$ Sortie (xml, mei, lilypond,… )\\
	Minimiser la distance entre le midi et la représentation en arbre.\\\\
	Le but du stage est d’améliorer qparse, un outil de transcription et d’écriture automatique de la batterie (entre autre)\\\\
	Le sujet de ce mémoire est de proposer une tâche de reconnaissance du regroupement des notes par les ligatures dans l’écriture de la batterie.\\\\
	Pour cela, nous utiliserons la logique des systèmes (selon la définition agostinienne).\\$\Rightarrow$ Motif répétitif de plusieurs instruments coordonnées accompagnés d’un texte varié joué par un autre instrument de la batterie.\\\\Nous partirons de propositions génériques de systèmes (environs trois systèmes dans différents style de batterie) que nous tenterons de détecter dans le jeu de données groove.\\\\
	Nous travaillerons aussi sur la détection de répétitions sur plusieurs mesures afin de pouvoir corriger des erreurs sur une des mesures qui aurait dû être identique au autres mais qui présente des différences.
	
\subsection*{Plan suivi dans le mémoire}
\begin{itemize}
	\item écouter le dataset groove
	\item observer la structure midi
	\item décrire la notation de la batterie et transcrire manuellement pour comparer l’input et l’output idéal d’un point de vu théorique.
	\item Détecter les systèmes dans le dataset.
	\item Faire des règles de réécriture (rewriting/simplification) et de séparation des voix (reconnaissance de systèmes)
\end{itemize}
	