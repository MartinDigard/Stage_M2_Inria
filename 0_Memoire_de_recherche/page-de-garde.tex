% ================================ Page du garde ==============================

\pdfbookmark[0]{Page de garde}{garde}
\thispagestyle{empty}

\begin{center}
  \begin{tabularx}{\textwidth}{m{10.3cm}m{4cm}}
	 \includegraphics[width = 3.9cm]{images/logos/logo-inalco.png} %% CG : 3.5cm au lieu de 3 cm
	&
        %% TODO: remplacer le logo du LIMSI par celui de la société ou
        %% du laboratoire où vous avez réalisé votre stage (si le
        %% sujet du mémoire de recherche fait suite à votre stage)
	 \includegraphics[width = 3.9cm]{images/logos/logo-inria.png} %% CG : 3.5cm au lieu de 3 cm
        \\ \hline
  \end{tabularx}
\end{center}

\begin{center}
\vspace{\stretch{1}}
% Permet de créer un espace vertical de longueur variable (\stretch) et de "poids" 1
{\Large \textbf{Institut National des Langues et Civilisations Orientales}}

\vspace{\stretch{1}}

{\normalsize Département Textes, Informatique, Multilinguisme}

\vspace{\stretch{2}}
\hrule
\vspace{\stretch{1}}
%% TODO: indiquez le titre de votre mémoire
{\LARGE \textbf{Titre du mémoire}}
\vspace{\stretch{1}}
\hrule

\vspace{\stretch{2}}

{\Huge \textsc{Master}}

\vspace{\stretch{1}}

{\LARGE \textsc{Traitement Automatique des Langues}}

\vspace{\stretch{1}}

{\normalsize \emph{Parcours~:}}

\vspace{\stretch{0.5}}

%% TODO: indiquez votre parcours
{\normalsize \emph{Ingénierie Multilingue}}

\vspace{\stretch{1}}

{\large par}

\vspace{\stretch{1}}

%% TODO: indiquez vos nom et prénom
\textbf{{\LARGE Martin \textsc{DIGARD}}}

\vspace{\stretch{2}}

{\normalsize \emph{Directeur de mémoire~:}}

\vspace{\stretch{0.5}}

%% TODO: indiquez le nom du/des directeur(s) de mémoire (enseignant
%% INaLCO qui supervise votre travail)
{\normalsize \emph{Damien NOUVEL}}

\vspace{\stretch{2}}

{\normalsize \emph{Encadrant~:}}

\vspace{\stretch{0.5}}

%% TODO: indiquez le nom du/des encadrant(s) de stage si votre mémoire
%% porte sur votre travail de stage
{\normalsize \emph{Florent JACQUEMARD}}

\vspace{\stretch{2}}

{\normalsize Année universitaire 2020/2021}

\end{center}

\cleardoublepage % pour laisser une page blanche au verso de la page de garde
