\subsection{Chaîne de traitement}
Chaîne de traitement :\\
if match(system\_parse\_tree(system + pitch), input\_midi\_parse\_tree)\\
\tab then voice\_split ;\\
\tab for voice in voice\_split:\\
\tab \tab simplication voice

\subsection*{Les sytèmes en batterie}
SYSTÈME ==> MOTIF (2-3 instruments sur 1-2 voix, joué en boucle) + TEXTE (1 instrument sur 1 voix, irrégulier)\\
ex : système afro-cubain, trois voix.
Proposition pour la détection de la direction des hampes et pour les ligatures (regroupement des notes et séparation des voix.)
\begin{itemize}
	\item \textbf{\textit{Les systèmes :}}\\
	$\Rightarrow$ Un système est la combinaison d’un ou plusieurs éléments qui jouent un rythme en boucle (système) et d’un autre élément qui joue un \textit{texte} rythmique variable mais respectant les règles propre au système (texte).\\
	Définition d’un système :\\
	
	En cas de système, les ligatures forment deux voies :
	\begin{itemize}
		\item Le texte ;
		\item Le système.
	\end{itemize}
	\textit{Mettre des exemples de différents systèmes.}
	\item \textbf{\textit{Les moulins :}}\\
	Lorsqu’il y a plus d’une voie, ils sont prioritaires pour les ligatures.\\
	\textit{Mettre des exemples.}\\
\end{itemize}
\subsection*{Gestion des silences et des têtes de notes}
Rythme tree (RT)\\
- Le symbole "-" est une continuation (liaison) mais pour une partition de
batterie, ça serait un silence par défaut sauf peut-être pour les ouvertures
de charley et éventuellement les cymbales ou un tom basse qui résonne.\\
$\Rightarrow$ Question de la (notation noir + silence) vs blanche.\\
$\Rightarrow$ On privilégierait (noir + silence) puisque les symboles « x » des cymbales ne
peuvent pas porter d’indication de durée dans la tête de notes.\\
Les 3 parties d’une note :
\begin{itemize}
	\item durée
	\item hampe
	\item tête de note (peut aussi indiquer la durée mais en batterie on évitera les blanches, etc.)
\end{itemize}
source : \url{https://fr.wikipedia.org/wiki/Note_de_musique}\\\\
\subsection*{Définition des têtes de notes et des hauteurs}
\subsubsection*{Proposition de définition d’un standard de départ}
Pour la transcriptions, nous proposons de choisir la base Agostini. La caisse claire centrale sur la portée est aussi centrale sur la batterie est elle est un élément qui conditionne la position des jambes (écart entre les pédales, etc.) ainsi que l’organisation des éléments en hauteur (toms, cymbales, etc.).
On pensera en terme de symétrie la répartition des éléments par rapport au point central que constitue la caisse claire.\\
Cette symétrie s’opère en trois dimensions :
\begin{itemize}
	\item Les hauteurs en terme de fréquences ;
	\item La hauteur physique des éléments :\\
	Du bas vers le haut : pédales, toms et caisse, cymbales
	\item L’ergonomie, qui hiérarchise l’importance des éléments sur la portée (caisse claire au centre, hh-pied et ride sont aux deux extrémités).
\end{itemize}
