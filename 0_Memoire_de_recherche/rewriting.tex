Basé sur \cite{jacquemard:hal-01134096} et sur \cite{jacquemard:hal-01403982}
\subsubsection*{Propositions de règles de ré-écriture}
Pour la plupart des instruments mélodiques, la liaison et le point sont les deux seules possibilités en cas d’équivalence rythmique. Mais la batterie offre un 3ème choix pour combler la distance rythmique entre deux notes : les silences.\\Les cymbales-crash et les ouvertures de charley constituent le seul cas qui exclut cette option. Le charley car ses ouvertures/fermetures sont presque toujours quantifiées. Les fermetures du charley sont notées soit par un silence (correspondant à une fermeture de la pédale), soit par un écrasement de l’ouverture par un autre coup de charley fermé, au pied ou à la main.
\begin{itemize}
\item 
Tie to dot devient Tie to note\&rest\\
par exemple : une noire liée à une autre noire devient une noire + un soupir.\\
On remplace donc la dernière note d’une liaison par sa valeur en silence.
Ceci ne concerne pas les ouvertures de charley, qui sont, avec les cymbales-crash, les seuls indications sonores dont nous considèrerons la durée.
\item 
Dans tous les cas, sauf ouverture de hh-pd, Une note sur un temps suivie d’une note en l’air sur le second temps seront écrites : noire sur le temps 1 puis silence + note pertinente sur le temps 2.
\item
Pour un charley ouvert qui déborde sur le temps d’après, le choix le plus pertinent est-il la note pointée ou la liaison ? Je pense que c’est la liaison car il marchera même dans les cas où le point ne marche pas (ex : mauvaise durée)
\item
une blanche sera écrite noir + soupir.\\
Voir exemples dans :
Stage\_M2\_Inria/2\_Transcriptions\_LilyPond/exemples\_rewriting
\end{itemize}
