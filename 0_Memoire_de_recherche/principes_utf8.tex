\chapter{Principes à suivre}

\section{Le sujet de votre mémoire}
Vous avez acquis, au cours de l'année 2015-2016, des compétences d'ingénieur-linguiste ; vous savez donc analyser un problème, proposer une méthodologie permettant d'arriver à une solution et montrer les limites de cette dernière. C'est cette démarche qui constituera le fil directeur de votre mémoire.

Ce travail devra être original et personnel. Le cadre de votre travail est naturellement la linguistique et, étant donné le diplôme que vous préparez, la linguistique appliquée, plutôt que théorique. Ceci ne veut néanmoins pas dire que vous ne devrez pas situer votre démarche à l'intérieur d'un cadre théorique, au contraire. On souhaite cependant que ce cadre serve d'appui à la création ou à la transformation d'outils, à la mise au point de méthodologies vous permettant de proposer un résultat.

Cela revient à dire que votre mémoire constitue une tentative de problématiser une approche méthodologique, de proposer une piste nouvelle, de comparer des méthodes, des outils, etc. Il contiendra en tout cas un état de l'art et s'appuiera sur une bibliographie précise et récente. L'état de l'art ne doit pas être déconnecté de la question traitée : on ne vous demande pas de \og{}faire un état de l'art pour faire un état de l'art\fg{} mais, au contraire, de montrer comment se situe votre travail par rapport à cet état de l'art.
Si votre sujet s'y prête, et afin d'en faciliter la réalisation, vous pouvez segmenter votre état de l'art en plusieurs parties ciblées à placer en tête des chapitres correspondant plutôt que d'écrire un chapitre consacré qui risque d'être généraliste et donc insuffisamment précis.

Vous devrez avoir choisi un sujet de mémoire à la mi-mai ou, à tout le moins, avoir réfléchi à des pistes sérieuses. Vous devrez vous assurer auprès d'un intervenant du TIM/ER-TIM que vous ne faites pas fausse route et que votre mémoire ne sera pas hors-sujet. Il s'agit d'éviter que vous ne traitiez un sujet dont les exigences techniques pourraient s'avérer supérieures à ce que vous croyez connaître. Le(s) stage(s) de fin d'études que vous devez entreprendre peu(t/vent) vous aider à affiner votre choix de sujet, mais vous devez garder à l'esprit que votre mémoire ne doit pas se confondre avec une description de votre stage. Notez bien que les rapports de stage ne sont pas pris en compte dans l'évaluation de votre Master.

Pour vous aider, vous pouvez consulter les meilleurs mémoires des années précédentes (et dont les résumés sont en ligne sur le site \url{www.er-tim.fr}). \'Evidemment, vous consulterez également les articles scientifiques liés à votre problématique : outre les connaissances que vous pourrez ainsi acquérir, cela vous permettra aussi de vous familiariser avec ce genre bien spécifique. Si vous ne trouviez pas de sujet vous permettant de mettre en pratique les connaissances acquises au cours de cette année, en fonction de vos goûts et attentes personnels ou professionnels, nous vous en proposerions un (consultez-nous, donc).

\section{L'encadrement du mémoire}
Vous avez toute latitude pour choisir, selon affinités, la/les personne(s) qui va/vont diriger vos recherches. Mais un/des intervenant(s) du TIM/ER-TIM figurera/ont nécessairement dans votre jury lors de la soutenance. Il faut donc nécessairement avoir pris contact avec ces personnes et s'assurer de leur collaboration. Si vous envisagez de faire une thèse ensuite, il est recommandé de solliciter un enseignant assimilé professeur ou habilité à diriger des recherches ou de mettre en place un co-encadrement en ce sens.

En règle générale, le TIM/ER-TIM souhaite, autant que faire se peut, que les personnes qui vous ont encadré lors de votre stage et qui ont pu vous conseiller pour la rédaction de votre mémoire, soient présentes lors de la soutenance. Elles apportent un complément d'information interne sur le stage et les conditions de réalisation du mémoire, éclairage qui peut être tout à fait pertinent.

Si vous rencontrez des problèmes et souhaitez poser des questions, il est impératif, dans un premier temps, de les formuler par courrier électronique plutôt que de venir immédiatement au TIM/ER-TIM, riche en compétences mais pauvre en personnel. Par ailleurs, vous ne devez pas envoyer par courrier électronique des centaines de pages à fin de re-lecture : lorsqu'une pré-version de votre travail vous semblera digne de relecture, déposez-la au TIM/ER-TIM, ou postez-la.

\section{L'évaluation du mémoire}
L'évaluation du mémoire est fonction de la qualité de votre travail écrit et de votre capacité à répondre aux questions, remarques, critiques qui peuvent vous être adressées pendant la soutenance. La qualité du travail écrit dépend de plusieurs critères, dont voici une liste non-exhaustive :
\begin{itemize}
\item votre mémoire forme-t-il un ensemble cohérent qui doit son unité à la volonté de répondre à une problématique bien définie ?
\item votre mémoire est-il réutilisable par une personne souhaitant faire un bilan de la problématique soulevée, tant du point de vue fond que forme (clarté de la bibliographie, description en annexe des outils utilisés avec liens aux sources, disponibilités des sources sur le CD-ROM d'accompagnement de votre mémoire, index permettant une consultation rapide, table des matières, pagination, etc.) ?
\item votre mémoire répond-t-il vraiment à l'objectif fixé au départ ? le titre de votre mémoire correspond-il vraiment au contenu ? les mots-clés qui seront mis en ligne sont-ils pertinents ?
\item votre mémoire met-il en valeur un angle de vue original sur un savoir-faire classique ?
\item votre mémoire parvient-il à mettre la théorie à l'épreuve ? \^Etes-vous capable de fournir des résultats, des exemples, un bilan d'expérience, des critères d'évaluation, une évaluation ?
\item la bibliographie doit être totalement normalisée, de façon à permettre une consultation aisée, les annexes contiendront un descriptif pratique et les références des outils utilisés, un échantillon des corpus utilisés et des programmes que vous avez écrits et, de manière générale, tout ce qui peut illustrer le travail réalisé. Attention, pour des raisons de place, vous ne devez évidemment pas présenter tous vos corpus et tous vos programmes en annexe, mais un simple échantillon. En revanche, corpus\footnote{Vérifiez toutefois que vous avez le droit de reproduire tout ou partie du corpus sur lequel vous aurez travaillé, en particulier pour les corpus de documents cliniques.} et programmes figureront impérativement et exhaustivement sur le CD fourni.
\end{itemize}

La qualité de votre prestation orale est importante. Vous devrez vous assurer, en particulier, que :
\begin{itemize}
\item vous savez vous affranchir du plan de votre mémoire mais vous devez néanmoins faire un bref résumé de la problématique car tous les membres du jury n'auront pas lu votre mémoire
\item vous donnez des exemples concrets des questions qui se sont posées et des solutions apportées, de façon à montrer que vous ne traitez pas le sujet de façon purement théorique
\item vous savez situer la problématique de votre mémoire par rapport aux travaux les plus connus et les plus récents sur la question
\item vous savez faire le lien entre les connaissances acquises au cours de l'année et la mise en pratique de ces connaissances lors de la réalisation du mémoire
\item vous savez répondre aux questions ou critiques qui vous sont soumises
\end{itemize}

\section{La démarche à suivre pour soutenir}
Trois semaines avant la date de soutenance, vous devez envoyer une version présentable de votre mémoire à votre encadrant et à l'équipe de formation, pour déterminer si le mémoire est soutenable. Vous devez remettre une version papier définitive de vos mémoires au moins 15 jours avant la soutenance.
\begin{itemize}
\item La soutenance pour la première session est fixée entre le 20 et le 24 juin 2016 (à préciser) pour ceux d'entre vous qui candidateraient à un contrat doctoral INaLCO (voir la procédure sur le site \url{www.inalco.fr}, le comité de sélection ayant lieu le 1 juillet 2016).
\item Pour la deuxième session (inscription en doctorat à l'INaLCO selon la procédure normale), la soutenance est fixée le 30 septembre 2016.
\item Pour la dernière session, la date de soutenance est fixée le 18 novembre 2016.
\end{itemize}

Vous devez déposer votre travail au moins deux semaines avant d'espérer soutenir. Il faut en effet qu'il soit lu, puis, si nécessaire, amendé et corrigé -- voire rejeté et réécrit -- de façon que la soutenance ne verse pas dans la critique systématique.

Au plus tard la veille de votre soutenance, vous aurez envoyé à \url{crim@inalco.fr} et à \url{sophie.urbaniak@inalco.fr} un résumé de votre mémoire de 10 lignes maximum ainsi que 5 mots-clés permettant de situer votre travail. Attention, ces informations sont destinées à être consultées et doivent donc être le reflet fidèle de votre travail final.

Une fois votre travail accepté, nous vous proposerons un ordre de passage pour la soutenance. Vous devrez fournir 3 exemplaires/support-papier et 3 exemplaires/support-électronique de votre mémoire (ces exemplaires sont destinés aux membres du jury et aux futurs étudiants). Sur le 4ème de couverture vous agraferez une enveloppe format 21-27 qui contiendra le CD correspondant à votre travail. Ce CD contiendra, outre la version électronique de votre mémoire, toutes les annexes ne pouvant figurer dans le mémoire pour des raisons de place : corpus, code source des outils utilisés, polices de caractères utilisées, code des programmes que vous avez élaborés.
