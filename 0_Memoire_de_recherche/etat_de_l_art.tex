\section{Introduction}

Dans ce chapitre, nous présentons les différentes avançées qui ont déjà eues lieues dans le domaine de la transcription.

\section{Contenu}
\textbf{Automatic music transcription : Challenges and future directions} \cite{article1}\\
(introduction\cite{article1})\\
\textbf{Les applications de l’AMT ont aussi de la valeur dans les domaine oraux qui manquent de partition (jazz, pop, (et donc batterie, note perso)}\\
(abstract \cite{article1})\\
Les différents travaux existant se préoccupent plus de la transcription à partir de l’audio en passant par le traitement du signal.\\
Les humains sont encore meilleur que les machines et la précisions à l’air d’avoir atteint sa limite.\\
Analyse des limites des méthodes courantes et identification des directions prometteuses.\\
Les modèles généraux utilisés ne traitent pas correctement la riche diversités des signaux musicaux.\\
2 moyens pour surmonter celà :
\begin{itemize}
	\item adapter algo pour des cas d’utilisations spécifiques.
	\item utiliser les approches semi-automatiques.
\end{itemize}
\textbf{La richesse des partitions musicales et des données audio correspondantes, désormais disponibles, constitue une source potentielle de données d'apprentissage, grâce à l'alignement forcé des données audio sur les partitions, mais l'utilisation à grande échelle de ces données n'a pas encore été tentée.}\\
D'autres approches prometteuses incluent l'intégration
d'informations provenant de plusieurs algorithmes et de différents aspects musicaux.
\section{Conclusion}
Conclusion de ce chapitre.