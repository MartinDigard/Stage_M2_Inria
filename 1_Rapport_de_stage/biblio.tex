\subsection*{Rassemblement d’ouvrages de référence sur la notation}
	\textbf{Méthodes Agostini :}
	\begin{itemize}
		\item solfèges rythmiques n° 1, 2, 3, 4 ;
		\item Méthodes de batterie n° 1, 2, 3, 4 ;
		\item Rythmiques binaires n° 1, 2 (J.-F. Juskowiak).\\
	\end{itemize}
	\textbf{Méthodes américaines :}\\
	Une dizaine de méthodes américaines ont été rassemblées car leur système de notation diffère légèrement de celui des méthodes agostini.\\
	
	\textbf{Autres ouvrages sur la batterie :}
	\begin{itemize}
		\item Une histoire de la batterie de jazz, Georges Paczynski.
		\item Rythme et geste, Georges Paczynski.\\
	\end{itemize}
	\textbf{Autres références sur la notation musicale :}
\begin{itemize}
	\item A. Danhauser, édition revue et augmentée
	Elle contient notamment des informations en plus l’écriture des métriques ;
	\item Behind Bars, the definitive guide to music notation, Elaine Gould.\\
\end{itemize}