\subsection*{Chaîne de traitement}
\begin{itemize}
	\item Reconnaître un motif (système) sur une mesure de l’input (un fichier midi représentant des données audios)\\ $\Rightarrow$ Motif (système) reconnu : true ou false
	\item Si true : 
	\begin{itemize}
		\item Séparer les voix (\textit{Règles établis par le système})
		\item Simplifier l’écriture de chaque voix (\textit{Règles établis par le système})\\
	\end{itemize}
\end{itemize}
%Chaîne de traitement :\\
%if match(system\_parse\_tree(system + pitch), input\_midi\_parse\_tree)\\
%\tab then voice\_split ;\\
%\tab for voice in voice\_split:\\
%\tab \tab simplication voice

\subsection*{Références pour l’évaluation}
1 - Transcription manuelle à partir de fichier midi et/ou wav d’une partition contenant des systèmes. Écriture des systèmes contenues dans la partition (arbres, séparation des voix, réécriture)