% !TEX root = 1_memoire.tex
QUOI ?

Ce mémoire de recherche, effectué en parallèle d’un stage à l’Inria dans le
cadre du master de traitement automatique des langues de l’Inalco, contient
une proposition originale ainsi que diverses contributions dans le domaine de
la transcription automatique de la musique. Les travaux qui seront exposés ont
tous pour objectif d’améliorer \textbf{qparse}, un outil de transcription
automatique de la musique, et seront axés spécifiquement sur le cas de la
batterie.

Nous parlerons de transcription musicale, en suivant des méthodes communes au
domaine du traitement automatique des langues (TAL) plutôt que directement de
langues naturelles, et nous parlerons aussi de génération automatique de
partitions de musique à partir de données audio ou symboliques. En considérant
que la musique à l’instar des langues naturelles est un moyen qui nous sert à
exprimer nos ressentis sur le monde et les choses, ce travail reposera sur une
citation de l’ouvrage de Danhauser \cite{danhauser} : « La musique s’écrit et
se lit aussi facilement qu’on lit et écrit les paroles que nous prononçons. »
L’exercice exposé dans ce mémoire nécessitera donc la manipulation d’un langage
musical qui peut être analysé à l’aide de théories formelles et d’outils
adéquats comme des grammaires (solfège, durées, nuances, volumes) et
soulèvera des problématiques qui peuvent être résolues par l’utilisation de
méthodes issues de l’informatique et de l’analyse des langues et des
langages.\\

POURQUOI ?
\begin{itemize}
	\item sujet traité : la batterie
	\item intérêt spécifique de la génération de partition de batterie comparativement
  au autres instrument
	\item patrimoine
	\item rapidité de génération (musicien ou enseignement)
	\item …\\
\end{itemize}

<flo>il faut revoir la fin, avec une description rapide du problème, de la
méthode suivie et des contributions suivi d'un petit plan par parties.</flo> 


COMMENT ?\\
$\to$ Problèmatique :\\
L’écriture musicale offre de nombreuses possibilités pour la transcription d’un
rythme donné. Le contexte musical ainsi que la lisibilité d’une partition pour
un batteur entraîné conditionnent les choix d’écriture. Reconnaître la
métrique principale d’un rythme, la façon de regrouper les notes par des
ligatures, ou simplement décider d’un usage pour une durée parmi les
différentes continuations possibles (notes pointées, liaisons, silences, etc.)
constituent autant de possibilités que de difficultés <dam>que de choix de
représentation à réaliser ?</dam>. De plus, la batterie est dotée d’une écriture
spécifique par rapport à la majorité des instruments.\\

$\to$ Méthodes :\\
$\to$ Contributions :\\
<louison>liste des contributions : donner une échelle, un point de comparaison, du
contexte, pour pouvoir mesurer l'importance de chaque contribution</louison>

La proposition principale de ce mémoire est basée sur la recherche de rythmes
génériques sur l’\textit{input}. Ces rythmes sont des \textit{patterns}
standards de batterie définis au préalable et accompagnés par les différentes
combinaisons qui leur sont propres. On les nomme systèmes (voir sections
\ref{systemes_methodes}, \ref{expe_theo}). L’objectif des systèmes est de fixer
des choix le plus tôt possible afin de simplifier le reste des calculs en
éliminant une partie d’entre eux. Ces choix concernent notamment la métrique
et les règles de réécriture.\\

La proposition ci-dessus a nécessité plusieurs sous-tâches :
\begin{itemize}
    \item une modélisation de la notation de la batterie
        (fusion de \ref{notation_batterie} et de
        \ref{modelisation_transcription}) qui était jusqu’à présent
        inexistante.
    \item plusieurs trancriptions manuelles dans le but d’analyser les contenus
        des fichiers MIDI et Audio (\ref{analyse_midi_audio}) et de faire des
        comparaisons de transcription avec des outils déjà existants
        \footnote{MuseScore3}.
    \item une partition de référence transcrite manuellement sur l’entièreté
        d’une performance du jeu de données afin de repérer les éléments
        importants pour la modélisation et de faire les liens entre les
        critères des données d’\textit{input} avec l’écriture finale
        (\ref{partition_ref}). Cette partition avait aussi pour objectif
        d’effectuer des tests et des évaluations.
    \item le passage au polyphonique en théorie et en implémentation impliquant
        la théorie sur la détection de l’identité de notes dans un Jam
        \footnote{groupe de notes rassemblées en raison d’un faible écart entre
            leur emplacements temporels} et l’implémentation de tests unitaires
            sur le traitement des Jams (\ref{jam_tests}).
    \item  la création de grammaires pondérées spécifiques à
        la batterie (\ref{gram_pond})\\
\end{itemize}

L’ensemble de ces sous-tâches a permis deux réalisations principales :
1) Obtenir des arbres de rythmes corrects en \textit{output} de qparse avec des
exemples courts proches de la partition de référence. \\
2) La création d’une expérimentation théorique d’un système \ref{expe_theo}
dont le but premier est de démontrer qu’elle est implémentable et applicable à
d’autres type de rythmes et dont le second objectif est de donner une méthode
de création d’un système à partir d’une partition.\\
Ces deux réalisations recouvrent une partie du chemin à parcourir puisque pour
effectuer des évaluations conséquentes sur résultat, la chaîne de traitement
doit être finie afin de pouvoir vérifier de manière empirique que les systèmes,
qui constituent ma contribution principale pour ce mémoire, ont permis
d’améliorer qparse pour la transcription automatique de la batterie.\\

PLAN\\
Nous présenterons le contexte (chapitre \ref{contexte}) suivi d’un état de l’art
(chapitre \ref{etat_de_l_art}) et nous définirons de manière générale le
processus de transcription automatique de la musique pour enfin étayer les
méthodes (chapitre \ref{methodes}) utilisées pour la transcription automatique
de la batterie.  Nous décrirons ensuite le corpus ainsi que les différentes
expérimentations menées (chapitre \ref{experimentations}). Nous concluerons par
une discussion sur les résultats obtenus et les pistes d’améliorations futures
à explorer. Les contributions apportées à l’outil qparse seront exposées dans
les chapitres \ref{methodes} et \ref{experimentations}.
