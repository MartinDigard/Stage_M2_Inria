Ce mémoire de recherche, effectué en parallèle d’un stage à l’Inria dans le
cadre du master de traitement automatique des langues de l’Inalco, contient
une proposition originale ainsi que diverses contributions dans le domaine de
la transcription automatique de la musique. Les travaux qui seront exposés ont
tous pour objectif d’améliorer qparse, un outil de transcription automatique de
la musique, et seront axés spécifiquement sur le cas de la batterie.

Nous parlerons de transcription musicale, en suivant des méthodes communes au
domaine du traitement automatique des langues (TAL) plutôt que directement de
langues naturelles, et nous parlerons aussi de génération automatique de
partitions de musique à partir de données audio ou symboliques. En considérant
que la musique, à l’instar des langues naturelles, est un moyen qui nous sert à
exprimer nos ressentis du monde et des choses, ce travail reposera sur une
citation de l’ouvrage de Danhauser \cite{danhauser}~ : « La musique s’écrit et
se lit aussi facilement qu’on lit et écrit les paroles que nous prononçons. »

L’exercice exposé dans ce mémoire nécessitera donc la manipulation d’un langage
codifié (solfège, durées, nuances, volumes) qui peut être analysé à l’aide de
théories formelles et d’outils adéquats comme des grammaires  et soulèvera des
problématiques qui peuvent être résolues par l’utilisation de méthodes issues
de l’informatique et de l’analyse des langues et des langages.\\

L’écriture musicale offre de nombreuses possibilités pour la transcription d’un
rythme donné. Le contexte musical ainsi que la lisibilité d’une partition pour
un batteur entraîné conditionnent les choix d’écriture. Reconnaître la
signature rythmique principale d’un rythme, la façon de regrouper les notes par
des ligatures, ou simplement décider d’un usage pour une durée parmi les
différentes continuations possibles (notes pointées, liaisons, silences, etc.)
constituent autant de possibilités que de choix de représentation à réaliser.
De plus, la batterie est dotée d’une écriture spécifique par rapport à la
majorité des instruments.\\

La proposition principale de ce mémoire est basée sur la recherche de rythmes
génériques sur l’\textit{input}. Ces rythmes sont des \textit{patterns}
standards de batterie définis au préalable et accompagnés par les différentes
combinaisons qui leur sont propres. On les nomme formes rythmiques (voir
section \ref{systemes_methodes}). L’objectif des formes rythmiques est de fixer
des choix le plus tôt possible afin de simplifier le reste des calculs en
éliminant une partie d’entre eux. Ces choix concernent notamment la signature
rythmique (voir section \ref{sign_rythm}) et les règles de réécriture (voir
section \ref{regles}).\\

La proposition ci-dessus a nécessité plusieurs sous-tâches~ :
\begin{itemize}
    \item une description de la notation de la batterie (voir
        \ref{notation_batterie} ainsi qu’une modélisation pour la transcription
        de la batterie \ref{modelisation_transcription}) qui était jusqu’à
        présent inexistante~ ;
    \item l’écriture de script lilypond \footnote{
        \url{http://lilypond.org/index.fr.html}} pour toutes les figures de ce
        mémoire qui correspondent à des partitions de batterie (voir la section
        \ref{tm})~ ; tous les codes lilypond sont disponibles sur \url{
        https://github.com/MartinDigard/Stage_M2_Inria}~ ;
    \item plusieurs trancriptions manuelles dans le but d’analyser les contenus
        des fichiers MIDI (voir section \ref{analyses_et_TM}) et de faire des
        comparaisons de transcriptions avec des outils déjà existants
        \footnote{MuseScore3}~ ;
    \item une partition de référence transcrite manuellement sur l’entièreté
        d’une performance du jeu de données afin de repérer les éléments
        importants pour la modélisation et d’établir le rapport entre les
        données d’\textit{input} et l’écriture finale (voir
        \ref{partition_ref})~ ; cette partition avait aussi pour objectif
        d’effectuer des tests et des évaluations~ ;
    \item  la création de grammaires hors-contexte pondérées spécifiques à
        la batterie (\ref{gram})~ ;
    \item le passage au polyphonique (théorie et implémentation de tests
        unitaires) impliquant la dissociation d’évènements MIDI simultanés
        ainsi que leur identification (voir section \ref{jam}. 
\end{itemize}

L’ensemble de ces sous-tâches a permis deux réalisations principales~ :
1) L’obtention d’arbres de rythmes corrects en \textit{output} de qparse avec
des exemples courts proches de la partition de référence (voir section
\ref{gram}).\\
2) La création d’une réécriture guidée par une forme rythmique
\ref{reecriture_guidee} dont le but premier est de démontrer qu’elle est
implémentable et applicable à d’autres types de rythmes et dont le second
objectif est de donner une méthode de création des formes rythmiques à partir
d’une partition.\\

Ces deux réalisations recouvrent une partie du chemin à parcourir puisque pour
effectuer des évaluations conséquentes sur résultats, la chaîne de traitement
doit être finie afin de pouvoir vérifier de manière empirique que les formes
rythmiques, qui constituent la proposition originale de ce mémoire, ont
permis d’améliorer qparse pour la transcription automatique de la batterie.\\

Nous présenterons le contexte (chapitre \ref{contexte}) suivi d’un état de
l’art (chapitre \ref{etat_de_l_art}) et nous définirons de manière générale le
processus de transcription automatique de la musique pour enfin étayer les
méthodes (chapitre \ref{methodes}) utilisées pour la transcription automatique
de la batterie.  Nous décrirons ensuite le corpus ainsi que les différentes
expérimentations menées (chapitre \ref{experimentations}). Nous concluerons par
une discussion sur les résultats obtenus et les pistes d’améliorations futures
à explorer. Les contributions apportées à l’outil qparse seront exposées dans
les chapitres \ref{methodes} et \ref{experimentations}.
