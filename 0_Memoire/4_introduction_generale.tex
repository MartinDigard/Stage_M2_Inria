% !TEX root = 1_memoire.tex
Ce mémoire de recherche, effectué en parallèle d’un stage à l’Inria dans le
cadre du master de traitement automatique des langues de l’Inalco, contient
une proposition originale ainsi que diverses contributions dans le domaine de
la transcription automatique de la musique. Les travaux qui seront exposés ont
tous pour objectif d’améliorer \textbf{qparse}, un outil de transcription
automatique de la musique, et seront axés spécifiquement sur le cas de la
batterie. Nous parlerons donc de transcription musicale, en suivant des
méthodes communes au domaine du traitement automatique des langues (TAL) plutôt
que directement de langues naturelles, et nous parlerons aussi de génération
automatique de partitions de musique à partir de données audio ou symboliques.
La musique et les langues naturelles sont deux moyens qui nous servent à
exprimer nos ressentis sur le monde et les choses : « La musique s’écrit et se
lit aussi facilement qu’on lit et écrit les paroles que nous prononçons. »
% Rendre la citation de danhauser explicite ?
\cite{danhauser}.

L’exercice exposé dans ce mémoire nécessitera la manipulation
d’un langage musical codifié par une grammaire (solfège, durées, nuances,
volumes) et soulèvera des problématiques concernées par les techniques du TAL.

L’écriture musicale offre de nombreuses possibilités pour la transcription d’un
rythme donné. Le contexte musical ainsi que la lisibilité d’une partition pour
un batteur entraîné conditionnent les choix d’écriture. Reconnaître la
métrique principale d’un rythme, la façon de regrouper les notes par les
ligatures, ou simplement décider d’un usage pour une durée parmi les
différentes continuations possibles (notes pointées, liaisons, silences, etc.)
constituent autant de possibilités que de difficultés. De plus, la batterie
est dotée d’une écriture spécifique par rapport à la majorité des
instruments. Il a donc fallu modéliser plusieurs critères concernant sa
représentation. Cette modélisation étaient jusqu’à présent inexistante.

%\florent{il faut revoir la fin, avec une description rapide du problème, 
         %de la méthode suivie et des contributions} %suivi d'un petit plan par parties.


% Voici la proposition principale de ce mémoire :


La proposition principale de ce mémoire est basée sur la recherche de rythmes
génériques sur l’\textit{input}. Ces rythmes sont des \textit{patterns}
standards de batterie définis au préalable et accompagnés par les différentes
combinaisons qui leur sont propres. On les nomme systèmes (voir sections
\ref{systemes_methodes}, \ref{expe_theo}). L’objectif des systèmes est de fixer
des choix le plus tôt possible afin de simplifier le reste des calculs en
éliminant une partie d’entre eux. Ces choix concernent notamment la métrique
et les règles de réécriture.

% Et voici les contributions apportées lors du stage :

La proposition ci-dessus a nécessité plusieurs sous-tâches dont une description
de la notation de la batterie (\ref{notation_batterie}) ainsi qu’une
modélisation pour la transcription de la batterie
(\ref{modelisation_transcription}).

Plusieurs trancriptions manuelles ont été effectuées afin d’analyser les
contenus des fichiers MIDI et Audio (\ref{analyse_midi_audio}) et de faire des
comparaisons de transcription avec des outils déjà existants
\footnote{MuseScore3}. Une partition entière a aussi était transcrite
manuellement afin de repérer les éléments importants pour la modélisation et
faire les liens entre les critères des données d’input avec \ref{partition_entiere}
L’ensemble de ces sous-tâches a permis la création expérimentation théorique
d’un système \ref{expe_theo}.

- Une fois proposition élaboré et verrouillé\\
- construire la chaîne jusqu’au bout\\
- nous avons pu la pousser jusqu’à la polyphonique en théorie et en
implémentation, ma contribution sur ce sujet étant la théorie sur la détection
de l’identité de notes dans un cluster (accord), l’implémentation de tests
unitaires sur les Jams (\ref{jam_tests}) et la création de grammaires pondérées
spécifiques à la batterie (\ref{gram_pond})


Les codes cpp sur la drum ont pu être construit en parti grâce aux travaux
réalisés dans ce mémoire de recherche.

% 
%	 Contributions sur la branche « distance » dans :
%	\begin{itemize}
%		\item qparselib/notes/cluster.md
%		\item qparselib/src/segment/import/ :\\
%		DrumCode hpp et cpp\\
%	\end{itemize}
%\end{itemize}

Nous présenterons le contexte (chapitre \ref{contexte}) suivi d’un état de l’art
(chapitre \ref{etat_de_l_art}) et nous définirons de manière générale le
processus de transcription automatique de la musique pour enfin étayer les
méthodes (chapitre \ref{methodes}) utilisées pour la transcription automatique
de la batterie.  Nous décrirons ensuite le corpus ainsi que les différentes
expérimentations menées (chapitre \ref{experimentations}). Nous concluerons par
une discussion sur les résultats obtenus et les pistes d’améliorations futures
à explorer. Les contributions apportées à l’outil qparse seront exposées dans
les chapitres \ref{methodes} et \ref{experimentations}.
