Ce mémoire de recherche, effectué en parallèle d’un stage à l’Inria dans le
cadre du master de traitement automatique des langues de l’Inalco, contient
une proposition originale ainsi que diverses contributions ayant toutes pour
objectif d’améliorer \textbf{qparse}, un outil de transcription automatique de
la musique sur sa capacité à transcrire la batterie. Nous ne parlerons donc pas
directement de langues naturelles, mais de l’écriture automatique de partitions
de musique à partir de données audio ou symboliques. La musique et les langues
naturelles sont deux moyens que nous servent à exprimer nos ressentis sur le
monde et les choses : « La musique s’écrit et se lit aussi facilement qu’on lit
et écrit les paroles que nous prononçons. » \cite{danhauser}.
Cette exercice nécessitera la manipulation d’un langage
musical codifié avec une grammaire (solfège, durées, nuances, volumes) et
soulèvera des problématiques concernées par les techniques du traitement
automatique des langues.\\
L’écriture musicale offre de nombreuses possibilités pour la transcription d’un
rythme donné. Le contexte musical ainsi que la lisibilité d’une partition pour
un batteur entraîné conditionnent les choix d’écriture. Reconnaître la
métrique principale d’un rythme, la façon de regrouper les notes par les
ligatures, ou simplement décider d’un usage pour une durée parmi les
différentes continuations possibles (notes pointées, liaisons, silences, etc.)
constituent autant de possibilités que de difficultés.\\\\
Voici la proposition de ce mémoire ainsi que les contributions apportées lors
du stage :
Rédiger entièrement la liste à puce qui suit.
\begin{itemize}
	\item Proposition principale : les systèmes (\ref{systemes_methodes},
		\ref{expe_theo}) :\\
	Recherche de rythmes génériques en amont dans la chaîne de traitement.\\
	$\Rightarrow$ L’objectif de fixer des choix le plus tôt possible afin de
	simplifier le reste des calculs en éliminant une partie d’entre eux. Ces
	choix concernent notamment la métrique et les règles de réécriture.
	\item Une description de la notation de la batterie
		(\ref{notation_batterie})
	\item Une modélisation de la transcription de la batterie
		(\ref{modelisation_transcription})
	\item Analyse MIDI-Audio (\ref{analyse_midi_audio})
	\item Transcription manuelle de partition \ref{partition_entiere}
	\item Expérimentation théorique d’un système \ref{expe_theo}
	\item Théorie et tests unitaires pour le passage au polyphonique
		(\ref{jam_tests})
	\item Création de grammaires pondérées pour la batterie (\ref{gram_pond})
	\item Contributions sur la branche « distance » dans :
	\begin{itemize}
		\item qparselib/notes/cluster.md
		\item qparselib/src/segment/import/ :\\
		DrumCode hpp et cpp\\
	\end{itemize}
\end{itemize}

Au lieu du paragraphe final : Nous présenterons dans un premier temps les
parallèles entre TAL et MIR, puis les spécificités de la notation pour la
batterie……… 

Nous présenterons le contexte suivi d’un état de l’art et nous définirons de
manière générale le processus de transcription automatique de la musique pour
enfin étayer les méthodes utilisées pour la transcription automatique de la
batterie, et nous présenterons les principales contributions apportées à
l’outil qparse. Nous décrirons ensuite le corpus ainsi que les différentes
expérimentations menées. Nous concluerons par une discussion sur les résultats
obtenus et les pistes d’améliorations futures à explorer.
