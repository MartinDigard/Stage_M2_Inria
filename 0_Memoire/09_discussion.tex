\section*{Introduction}
Dans ce chapitre, nous discuterons sur la pertinence de l’ensemble des choix qui ont été faits. Nous ferons un bilan des différentes avancées qui ont été faites ou non et nous tenterons d’en expliquer la ou les raisons.
\section{Travaux réalisés}
\textit{Faire une auto-critique des travaux réalisés.}
\subsection{Développer la notation}
\subsection{La modélisation}
\subsection{Le jeu de système}

\section{Travaux non-réalisés}
\textit{Expliquer pourquoi ces travaux n’ont pas pu être réalisés.}
\begin{itemize}
	\item implémenter un pattern…\\
	$\Rightarrow$ manque de temps ?\\
	\item La partie résultat est manquante car :\\
	$\Rightarrow$ Sujet très difficile ;\\
	$\Rightarrow$ Matcher les motifs peut être fait ultérieurement ;\\
	\tab Mais ce travail aurait été indispensable pour obtenir une quan-\tab tité de résultats qui justifieraient une évaluation automatique \tab permettant de faire des graphiques.\\
	\item L’évaluation fut entièrement manuelle car :\\
	$\Rightarrow$ Très dure automatiquement : il faut comparer 2 partitions (réf \tab VS output)
\end{itemize}
\section{Travaux futures}
\begin{itemize}
	\item Le ternaire jazz (voir expérience 2)
	\item Reconnaissance d’un motif sur le MIDI\\
	Reconnaître un motif (système) sur une mesure de l’input (un fichier midi représentant des données audios)\\
	$\Rightarrow$ Motif (système) reconnu : true ou false\\
	Si true :\\
	- Choisir la grammaire correspondante ;\\
	- Parser le MIDI ;\\
	- Appliquer les règles de réécritures (Séparation des voix et simplification)
	\item Nous travaillerons aussi sur la détection de répétitions sur plusieurs mesures afin de pouvoir corriger des erreurs sur une des mesures qui aurait dû être identique aux autres mais qui présente des différences.
	\item dans quelle catégorie mettre le shuffle ?
\end{itemize}
\section*{Conclusion}
Sujet passionnant mais difficile. Obtenir la totalité des critères pour le mémoire n’aurait pas pu être fait sans bâcler. Une base solide spécifique à la batterie a été générée. Elle sera un bon point de départ pour les travaux futurs dont plusieurs propositions sont énoncés dans le présent document.