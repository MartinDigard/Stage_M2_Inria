Ce mémoire de recherche, effectué en parallèle d’un stage à l’Inria dans le cadre du master de traitement automatique des langues de l’Inalco, contient une proposition d’amélioration de Qparse, un outil de transcription et d’écriture automatique de la musique sur sa capacité à transcrire la batterie. Nous ne parlerons donc pas directement de langues naturelles, mais de l’écriture automatique de partitions de musique à partir de données audios. Cette exercice nécessitera la manipulation d’un langage musical codifié avec une grammaire (solfège, durées, nuances, volumes) et soulèvera des problématiques concernées par les techniques du traitement automatique des langues.\\
La batterie est un instrument récent qui s’est longtemps passé de partition. En effet pour un batteur, la qualité de lecteur lorsqu’elle était nécessaire, résidait essentiellement dans sa capacité à lire les partitions des autres instrumentistes (par exemple, les grilles d’accord et la mélodie du thème de jazz) afin d’improviser un accompagnement approprié que personne ne pouvait écrire pour lui à sa place. Les partitions de batterie sont arrivées par nécessité avec la pédagogie et l’émergence d’école de batterie partout dans le monde. La musique assistée par ordinateur (MAO), a elle aussi largement contribué à l’expansion des partitions de batterie car l’autonomie musiciens pour écouter le résultats de leur compositions sans nécessairement avoir les musiciens pour jouer leur a permis d’écrire ce qu’il voulait entendre pour leur futur batteur.\\
L’écriture musicale offre de nombreuses possibilités pour la transcription d’un rythme donné. Le contexte musical ainsi que la lisibilité d’une partition pour un batteur entraîné conditionnent les choix d’écritures. Reconnaître la métrique principale d’un rythme, la façon de regrouper les notes par les ligatures, ou simplement décider d’un usage pour une durée parmi les différentes continuations possibles (notes pointées, liaisons, silences, etc.) constituent autant de possibilités que de difficultés.\\
Nous proposons de rechercher des rythmes génériques (\textit{motifs}) en amont dans la chaîne de traitement. Les \textit{motifs} sont prédéfinis avec des combinaisons possibles (\textit{gammes}) qui leur sont associées. Ces \textit{motifs} et leur \textit{gammes} respectives sont appelés \textit{systèmes}. L’usage des \textit{systèmes} a pour objectif de fixer des choix le plus tôt possible dans la chaîne de traitement afin de simplifier le reste des calculs en éliminant une partie d’entre eux. Ces choix concernent notamment la métrique, la séparation des voix ainsi que les règles de réécriture.\\
Nous dresserons dans une première partie, un état de l’art et nous définirons de manière générale le processus de transcription automatique de la musique pour enfin étayer les méthodes utilisées pour la présente proposition. Dans une seconde partie, Le corpus sera présenté ainsi que les différentes expérimentations menées. Nous concluerons par une discussion sur les résultats obtenus et les pistes d’améliorations futures à explorer.