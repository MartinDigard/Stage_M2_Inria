%\section{Contenu}
L'objectif de la transcription automatique de la musique (AMT) \cite{article1} est de convertir la performance d'un musicien en notation musicale - un peu comme la conversion de la parole en texte dans le traitement du langage naturel. Elle est considérée comme l'un des problèmes de recherche les plus anciens et les plus difficiles dans le domaine de la recherche d'information musicale (MIR).\\
Le cas de la transcription de la batterie (DT) est très particulier puisqu'il s'agit d'instruments sans hauteur, d'événements avec (presque) aucune durée et de notations spécifiques. Il a été la source de nombreuses études MIR, voir \cite{8350302} pour un aperçu. La plupart de ces travaux se concentrent sur des méthodes de calcul pour la détection d'événements sonores de batterie à partir de signaux acoustiques, et sur l'extraction de caractéristiques de bas niveau telles que la classe d'instrument et le moment de l'apparition du son (peak picking). Cependant, très peu d'entre eux ont abordé la tâche de générer une notation musicale (rythmique) lisible à partir des caractéristiques ci-dessus, une étape cruciale dans un contexte musical et loin d'être triviale.\\\\
\textbf{Automatic music transcription : Challenges and future directions} \cite{article1}\\
(introduction\cite{article1})\\
\textbf{Les applications de l’AMT ont aussi de la valeur dans les domaine oraux qui manquent de partition (jazz, pop, (et donc batterie, note perso)}\\
(abstract \cite{article1})\\
Les différents travaux existant se préoccupent plus de la transcription à partir de l’audio en passant par le traitement du signal.\\
Les humains sont encore meilleurs que les machines et la précisions à l’air d’avoir atteint sa limite.\\
Analyse des limites des méthodes courantes et identification des directions prometteuses.\\
Les modèles généraux utilisés ne traitent pas correctement la riche diversités des signaux musicaux.\\\\
2 moyens pour surmonter cela :
\begin{itemize}
	\item Adapter les algorithmes pour des cas d’utilisations spécifiques.
	\item Utiliser les approches semi-automatiques.\\
\end{itemize}
\textbf{La richesse des partitions musicales et des données audio correspondantes, désormais disponibles, constitue une source potentielle de données d'apprentissage, grâce à l'alignement forcé des données audio sur les partitions, mais l'utilisation à grande échelle de ces données n'a pas encore été tentée.}\\\\
D'autres approches prometteuses incluent l'intégration d'informations provenant de plusieurs algorithmes et de différents aspects musicaux.\\
\textit{Voir : A Review of Automatic Drum Transcription}\cite{8350302}