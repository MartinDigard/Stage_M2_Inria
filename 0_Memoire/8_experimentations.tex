\section*{Introduction}
Dans ce chapitre, nous présenterons le jeu de données et les analyse MIDI-Audio
et transcriptions manuelles.

Problèmatique :

choix d’un outil de transcription ?

passage au polyphonique indispensable pour la suite du travail
et pour l’usage des formes rythmiques.

Finir la chaîne de traitement indispensable pour obtenir des résultats chiffrés
possible à évaluer.

Nous présenterons mes trois contributions principales :
\begin{itemize}
    \item le code lilypond normalisé pour la transcription de la batterie avec
        la notation de type agostini.
    \item les différentes étapes de résolution du passage au polyphonique.
    \item l’expérimentation d’un forme rythmique implémentable qui devra être
        utilisé comme base de connaissances pour augmenter la rapidité et la
        qualité en sortie de Qparse et comme une méthode de création de
        nouvelles formes rythmiques.
\end{itemize}

Enfin, nous finirons par une discussion sur les avancées réalisées dans ce
travail, la pertinence des choix qui ont été faits et les moyens d’évaluer les
résultats potentiels.

\section{Le jeu de données}
Nous avons utilisé le Groove MIDI
Dataset\footnote{\url{https://magenta.tensorflow.org/datasets/groove}}
\cite{groove2019} (GMD) qui est un jeu de données mis à disposition par Google
sous la licence Creative Commons Attribution 4.0 International (CC BY 4.0).\\
Le GMD est composé de 13,6 heures de batterie sous forme de fichiers MIDI et
audio alignés. Il contient 1150 fichiers MIDI et plus de 22 000 mesures de
batterie dans les styles les plus courants et avec différentes qualités de jeu.
Tout le contenu a été joué par des humains sur la batterie électronique Roland
TD-11 (figure \ref{electro_drums}).
\begin{figure}[h]
	\centering
	\includegraphics[height=35mm, width=60mm]
    {z_images/4_experimentations/0_groove/0_roland.png}\ \ 
	\includegraphics[height=35mm, width=60mm]
    {z_images/4_experimentations/0_groove/1_roland.png}
	\caption{Batterie électronique}
	\label{electro_drums}
	\textit{Source :} \url{https://www.youtube.com/watch?v=BX1V_IE0g2c}
\end{figure}

Autres critères spécifiques au GMD :
\begin{itemize}
	\item Toutes les performances ont été jouées au métronome et à un tempo
        choisi par le batteur.
	\item 80\% de la durée du GMD a été joué par des batteurs professionnels
        qui ont pu improviser dans un large éventail de styles. Les données
        sont donc diversifiées en termes de styles et de qualités de jeu
        (professionnel ou amateur).
	\item Les batteurs avaient pour instruction de jouer des séquences de
        plusieurs minutes ainsi que des fills\footnote{Un \textit{fill} est une
        séquence de relance dont la durée dépasse rarement 2 mesures. Il est
        souvent joué à la fin d’un cycle pour annoncer le suivant.}
	\item Chaque performance est annotée d’un style (fourni par le batteur),
        d’une signature rythmique et d’un tempo ainsi que d’une identification
        anonyme du batteur.
	\item Il a été demandé à 4 batteurs d’enregistrer le même groupe de 10
        rythmes dans leurs styles respectifs. Ils sont dans les dossiers
        eval-session du GMD.
	\item Les sorties audio synthétisées ont été alignées à 2 ms près sur leur
        fichier MIDI.
\end{itemize}

\subsection*{Format des données}
Le Roland TD-11 enregistre les données dans des fichiers MIDI et les divise en
plusieurs pistes distinctes :
\begin{itemize}
	\item une pour le tempo et l’indication de mesure ;
	\item une pour les changements de contrôle (position de la pédale de
        charley) ;
	\item une pour les notes.\\
\end{itemize}
Les changements de contrôle sont placés sur le canal 0 et les notes sur le
canal 9 (qui est le canal canonique pour la batterie).\\
Pour simplifier le traitement de ces données, ces trois pistes ont été
fusionnées en une seule piste qui a été mise sur le canal 9.

\section{Analyses et transcriptions manuelles}
\label{analyses_et_TM}
Ces analyses ont été faites dans le cadre de transcriptions manuelles à partir
de fichiers MIDI et Audio du GMD.

\subsection*{Comparaisons de transcriptions}
Pour les comparaisons de transcriptions, les transcriptions manuelles (TM) ont
été éditées à l’aide de Lilypond\footnote{\url{http://lilypond.org/}} ou
MuseScore\footnote{\url{https://musescore.com/}} et les transcriptions
automatiques (TA) ont toutes été générées par import d’un fichier MIDI dans
MuseScore.

\subsubsection{Exemple d’analyse 1}
\begin{figure}[h]
\centering
Transcription manuelle $\Rightarrow$ Transcription automatique
\includegraphics[height=20mm, width=50mm]{
z_images/4_experimentations/1_analyses/0_drummer1_session3/1_manuelle.png}
\ \ \ \ 
\includegraphics[height=20mm, width=45mm]{
z_images/4_experimentations/1_analyses/0_drummer1_session3/0_musescore.png}
\end{figure}
\begin{itemize}
	\item Erreur d’indication de mesure (3/4 au lieu de 4/4) ;
	\item Les silences de la mesure 1 de la TA sont inutilement surchargés ;
	\item La noire du temps 4 de la mesure 1 de la TM est devenue les deux
        premières notes (une double-croche et une croche) d’un triolet sur le
        temps 1 de la mesure 2 de la TA.
\end{itemize}

\subsubsection{Exemple d’analyse 2}
\begin{figure}[h]
\centering
\tab Transcription manuelle $\Rightarrow$ Transcription automatique\\
\includegraphics[height=20mm, width=13mm]{
z_images/4_experimentations/1_analyses/0_drummer1_session3/5_manuelle.png}
\ \ \ \ 
\includegraphics[height=20mm, width=13mm]{
z_images/4_experimentations/1_analyses/0_drummer1_session3/4_musescore.png}
\end{figure}
\begin{itemize}
	\item Les doubles croches ont été interprétées en quintolet
	\item La deuxième double-croche est devenue une croche.
\end{itemize}

\subsubsection{Exemple d’analyse 3}
\begin{figure}[h]
\centering
Transcription manuelle $\Rightarrow$ Transcription automatique
\includegraphics[height=24mm, width=50mm]{
z_images/4_experimentations/1_analyses/0_drummer1_session3/3_manuelle.png}
\ \ \ \ 
\includegraphics[height=25mm, width=55mm]{
z_images/4_experimentations/1_analyses/0_drummer1_session3/2_musescore.png}
\end{figure}

\begin{itemize}
	\item Les grosses-caisses, les charleys et les caisses-claires ont été
        décalés d’un temps vers la droite.
	\item Les toms basses des temps 1 et 2 de la mesure 2 de la TM ont été
        décalés d’une double croche vers la droite dans la TA.
	\item La première caisse-claire de la mesure 1 devient binaire dans la TA
        alors qu’elle appartenait à un triolet dans la TM.
	\item Le triolet de tom-basse du temps 4 de la mesure 2 de la TA n’existe
        pas la TM.\\
\end{itemize}

\subsubsection{Exemple d’analyse 4}
\tab \tab Transcription manuelle $\Rightarrow$ Transcription automatique
\begin{figure}[h]
\centering
\includegraphics[height=19mm, width=50mm]{
z_images/4_experimentations/1_analyses/1_drummer1_session1/3_manuelle.png}
\ \ \ \ 
\includegraphics[height=19mm, width=70mm]{
z_images/4_experimentations/1_analyses/1_drummer1_session1/2_musescore.png}
\end{figure}

Sur le temps 4 de la mesure 1, la deuxième croche a été transcrite d’une
manière excessivement complexe!

\subsubsection{Exemple d’analyse 5 (flas)}
\label{flas}
Transcription manuelle\\
\includegraphics[height=25mm, width=95mm]{
z_images/4_experimentations/1_analyses/2_flas/0_124_funk_95_fill_4-4.png}\\
Transcription automatique\\\\
\includegraphics[height=20mm, width=130mm]{
z_images/4_experimentations/1_analyses/2_flas/1_124_funk_95_fill_4-4.png}\\

\begin{itemize}
	\item Le premier fla est reconnu comme étant un triolet contenant une
        quadruple croche suivie d’une triple croche au lieu d’une seule note
        ornementée.
	\item Le deuxième fla est reconnu comme étant un accord.
	\item Les deux double en contre-temps sur le temps 4 de la TM sont mal
        quantifiée dans la TA. 
	\item La TA ne reconnaît qu’une mesure quand la TM en transcrit deux. En
        effet, la TA a divisé par deux la durée des notes afin de les faire
        tenir dans une mesure à 4 temps dont les unités de temps sont les
        noires. Par exemple, le soupir du temps 2 de la TM devient un
        demi-soupir sur le contre-temps du temps 1 dans la TA. Ou encore, la
        noire (pf, voir le tableau \ref{pitchs_instru}) sur le temps 1 de la
        mesure 2 de la TM suivie d’un demi-soupir devient une croche pointée
        sur le temps 3 de la TA.
	\item Autre problème : certaines têtes de notes sont mal attribuées. Par
        exemple, le charley ouvert en contre-temps sur le temps 2 de la mesure
        2 de la TM devrait avoir le même symbole sur la TA. Idem pour les
        cross-sticks.
\end{itemize}

\subsubsection{Conclusion d’analyse}
Ces analyses ont montré la difficulté pour un logiciel comme MuseScore d’offrir
une partition lisible. Les raisons sont le fait que les fichiers MIDI ne sont
pas encore quantifiés mais aussi qu’il n’y a pas de reconnaissance de la forme 
du rythme impliquant sa position dans la mesure. Cette reconnaissance pourrait
permettre de rectifier les problèmes de signature rythmique ainsi que les
problèmes de décalage de temps. La reconnaissance de la forme du rythme
permettrait aussi de supprimer les aberrations du type de celle de l’exemple
d’analyse 4, puisque l’erreur sur cet exemple serait reconnue comme un élément
qui ne rentre pas dans le cadre de la forme de rythme en question. La dernière
raison qui rend le travail difficile est l’identification des flas, comment
savoir si deux notes jouées très proches sont :
\begin{itemize}
    \item séparées et rapides,
    \item mal jouées à l’unisson (accord),
    \item ou forment un fla ?
\end{itemize}


\subsection*{Transcription de partition}
\label{partition_entiere}
La figure \ref{partition_ref} est la transcription manuelle des fichiers
\textit{004\_jazz-funk\_116\_beat\_4-4.mid} et
\textit{004\_jazz-funk\_116\_beat\_4-4.wav} du GMD.

Cette transcription a été entièrement faite avec Lilypond (voir le code
lilypond sur le git \url{https://github.com/MartinDigard/Stage_M2_Inria}). Il
s’agit d’une partition d’un 4/4 binaire dont le fichier MIDI est annoncé dans
le GMD de style «jazz-funk» probablement en raison de la ride de type shabada
rapide (le ternaire devient binaire avec la vitesse) combiné avec l’after-beat
de type rock (caisse-claire sur les deux et quatre).

La transcription manuelle de la partition de la figure \ref{partition_ref} et
l'analyse d'autre fichiers MIDI (voir section \ref{analyses_et_TM}) m’ont mené
aux observations suivantes :
\begin{itemize}
	\item Vélocité inférieure à 40 : ghost-note ;
	\item Vélocité supérieure à 90 : accent ;
	\item Pas d’intention d’accent ni de ghost-note pour une vélocité entre 40
        et 89 ;
	\item Les accents et les ghosts-notes ne sont significatifs ni pour les
        instruments joués au pied, ni pour les cymbales crash.\\
	En effet, certaines vélocités en dessous de 40 étant détectées et inscrites
    dans les données MIDI sont dues au mouvement du talon du batteur qui bat la
    pulsation sans particulièrement jouer le charley. Ce mouvement est perçu
    par le capteur de la batterie électronique mais le charley n’est pas joué.
	\item Au final, j’ai relevé les ghost-notes et les accents pour la caisse
        claire ainsi que les accents pour les toms et les cymbales rythmiques
        (charley et ride).
\end{itemize}


\subsubsection{Conclusion sur les transcriptions manuelles}
La transcription des données audio et MIDI contenues dans ces fichiers a permis
une analyse plus approndie des critères à relever pour chaque évènement MIDI et
de la manière de les considérer dans un objectif de transcription en partition
lisible pour un musicien (Voir la section \ref{modelisation_transcription}).


\section{Transcription polyphonique par parsing}
Les Jams permettent de passer du monophonique au polyphonique.
On regroupe tous les évènements qu’on estime être à des dates suffisamment
proches dans des clusters que nous avons appelés « Jam ».
Un Jam représente une séquence de points successifs dans un segment d'entrée
avec des informations supplémentaires sur leur rôle respectif, en supposant
qu'ils sont tous alignés à la même date lors de la quantification. Un Jam est
caractérisé par l'index de son premier point dans le segment d'entrée et sa
longueur (nombre de points).
Une fois dans les Jams, les évènements sont identifiés selon les critères
suivants :
\begin{itemize}
    \item Ignored, les offsets sont ignorés ;
    \item Note, si c’est une note simple ;
    \item Rest, si c’est un silence ;
    \item GraceNote, si c’est un fla ;
    \item Error.
\end{itemize}

\section{Réécriture guidée par une forme rythmique}
\label{reecriture_guidee}
La démonstration qui suit est basée sur la partition de référence de la figure
\ref{partition_ref} puisque la forme rythmique qui sera utilisée en est
directement extraite.\\

Nous allons montrer :
\begin{itemize}
    \item la composition de cette forme rythmique ;
    \item son état finale, c’est à dire toutes les combinaisons entièrement
        écrites en notation correcte sur partition ;

        $\Rightarrow$ cela constituera une référence pour la réécriture ;
    \item un exemple de transformation de la forme rythmique en arbre de
        rythme ;
    \item l’application de la séparation des voix sur cet exemple basé sur la
        référence citée précédemment (la forme rythmique en question) ;\\
        $\Rightarrow$ l’arbre de départ sera alors séparé en autant d’arbres
        qu’il y a de voix (deux arbres pour cette forme rythmique) ;
    \item les règles de simplification propres à la forme rythmique dont nous
        parlons. 
\end{itemize}
L’objectif de cette démonstration est de montrer comment un jeu de plusieurs
formes rythmiques pourrait être implémenter dans le cadre d’une approche
dictionnaire.

\subsection*{Motifs et gammes}
\begin{figure}[h]
\centering
\includegraphics[height=43mm, width=40mm]{
z_images/4_experimentations/2_reecriture_guidee/0_motifs_4-4_binaires.png}
\includegraphics[height=55mm, width=85mm]{
z_images/4_experimentations/2_reecriture_guidee/1_gammes_4-4_binaires.png}
\caption{Motifs et gammes}
\label{motifs_gammes}
\end{figure}

\subsubsection{Motifs}
À partir de la partition de référence, les deux motifs de la figure
\ref{motifs_gammes} peuvent être systématisés. Le motif 1 est joué du début
jusqu’à la mesure 18 avec des variations et des fills et le motif 2 est joué de
la mesures 23 à la mesure 28 avec des variations. Ces deux motifs sont très
classiques et pourront être détectés dans de nombreuses performances.\\

\subsubsection{Gammes}
Les gammes de la figure \ref{motifs_gammes} étayent toutes les combinaisons
d’un motif en 4/4 binaires jusqu’aux doubles croches.\\
Les lignes 1 et 2 traitent les croches. La ligne 1 a 2 mesures dont la première
ne contient que des noires et la deuxième que des croches en contre-temps. Ces
deux possibilités sont combinées de manière circulaire dans les 3 mesures de la
deuxième ligne.\\
Les lignes 3, 4 et 5 traitent les doubles-croches. La ligne 3 a 2 mesures dont
la première ne contient que des croches et la deuxième que des doubles-croches
en contre-temps. Ces deux possibilités sont combinées de manière circulaire
dans les lignes 4 et 5 qui contiennent chacunes 3 mesures.

\subsection*{Formes rythmiques — motifs et gammes combinés}
Pour la suite de cette démonstration, je utiliserai le motif 1 de la
figure \ref{motifs_gammes}.<dam>à commenter un peu plus, notamment pour dire si
la combinaison est faite automatiquement ou non</dam> 
\begin{figure}[h]
\centering
\includegraphics[height=75mm, width=85mm]{
z_images/4_experimentations/2_reecriture_guidee/2_systeme_4-4_binaire.png}
\caption{Partition d’un forme rythmique en 4/4 binaire}
\label{sys_binaire}
\end{figure}
\subsection*{Représentation de la forme rythmique en arbres de rythmes}
\label{demo_sys}
\begin{figure}[h]
	\centering
	\resizebox{350pt}{!} {
		\Tree[.Motif\ 1\ +\ gamme\ 1a
		[.Mesure\ 1
		[.Temps\ 1 [rd\\bd ][ [rd\\pf ][rd ]]]
		[.Temps\ 2 [rd\\cc ][ [rd\\pf ][rd ]]]
		[.Temps\ 3 [rd\\bd ][ [rd\\pf ][rd ]]]
		[.Temps\ 4 [rd\\cc ][ [rd\\pf ][rd ]]] ]
		[.Mesure\ 2
		[.Temps\ 1 [rd ][ [rd\\bd\\pf ][rd ]]]
		[.Temps\ 2 [rd\\cc ][ [rd\\bd\\pf ][rd ]]]
		[.Temps\ 3 [rd ][ [rd\\bd\\pf ][rd ]]]
		[.Temps\ 4 [rd\\cc ][ [rd\\bd\\pf ][rd ]]] ]]}
	\caption{Arbre de rythme — forme rythmique}
	\label{arbre_sys}
\end{figure}
L’arbre de la figure \ref{arbre_sys} servira de base pour le suite de
l’expérimentation. Comme indiqué à la racine de l’arbre, il représente la
première ligne de la figure \ref{sys_binaire}. Même si cet arbre représente
parfaitement le rythme concerné, il manque des indications de notation telles
que les voix spécifiques à chaque partie du rythme ainsi que les choix
d’écriture pour les distances qui séparent les notes de chaque voix entre elles
en termes de durée.

\subsection*{Réécriture — séparation des voix et simplification}
\subsubsection{La séparation des voix}
Ainsi l’arbre syntaxique de départ est divisé en autant d’instruments qui le
constituent et les voix seront regroupées en suivant les régles du forme
rythmique.
\begin{figure}[h]
	\centering
	\resizebox{350pt}{!} {
		\Tree[.Motif\ 1\ +\ Gamme\ 1a
		[.Mesure\ 1
		[.Temps\ 1 [rd ][ [rd ][rd ]]]
		[.Temps\ 2 [rd\\cc ][ [rd ][rd ]]]
		[.Temps\ 3 [rd ][ [rd ][rd ]]]
		[.Temps\ 4 [rd\\cc ][ [rd ][rd ]]] ]
		[.Mesure\ 2
		[.Temps\ 1 [rd ][ [rd ][rd ]]]
		[.Temps\ 2 [rd\\cc ][ [rd ][rd ]]]
		[.Temps\ 3 [rd ][ [rd ][rd ]]]
		[.Temps\ 4 [rd\\cc ][ [rd ][rd ]]] ]]}
	\caption{Arbre de rythme — voix haute}
	\label{voix_haute}
\end{figure}\\
La voix haute (figure \ref{voix_haute}) regroupe la ride et la caisse-claire
sur les ligatures du haut.
\begin{figure}[h]
	\centering
	\resizebox{350pt}{!} {
		\Tree[.Motif\ 1\ +\ Gamme\ 1a
		[.Mesure\ 1
		[.Temps\ 1 [bd ][ [pf ][t ]]]
		[.Temps\ 2 [t ][ [pf ][t ]]]
		[.Temps\ 3 [bd ][ [pf ][t ]]]
		[.Temps\ 4 [t ][ [pf ][t ]]] ]
		[.Mesure\ 2
		[.Temps\ 1 [t ][ [bd\\pf ][t ]]]
		[.Temps\ 2 [t ][ [bd\\pf ][t ]]]
		[.Temps\ 3 [t ][ [bd\\pf ][t ]]]
		[.Temps\ 4 [t ][ [bd\\pf ][t ]]] ]]}
	\caption{Arbre de rythme — voix basse}
	\label{voix_basse}
\end{figure}\\
La voix basse (figure \ref{voix_basse} regroupe la grosse-caisse et le charley
au pied sur les ligatures du bas.
\subsubsection{Les règles de simplifications}
L’objectif des règles de simplifications est de réécrire les écarts de durées
qui séparent les notes d’une manière appropriée pour la batterie et qui soit la
plus simple possible. Les ligatures relient les notes d’un temps entre elles
afin de rendre la pulsation visuelle).\\\\
Pour les figures ci-dessous :
\begin{itemize}
	\item x = une note ;
	\item r = un silence ;
	\item t = une continuation (point ou liaison)
\end{itemize}
\begin{figure}[h]
	\centering
	\resizebox{50pt}{!} {
		\Tree[.1/4 [x ][ [x ][t ]] ]
	}\ \ \ \ \ $\Rightarrow$\ \ \ \ \
	\resizebox{30pt}{!} {
		\Tree[.1/4 [x ][x ] ]
	}\\
\includegraphics[height=10mm, width=25mm]{
z_images/4_experimentations/2_reecriture_guidee/simplification_0.png}\ \ \ \ \ 
$\Rightarrow$\ \ \ \ \
\includegraphics[height=10mm, width=20mm]{
z_images/4_experimentations/2_reecriture_guidee/simplification_1.png}
	\caption{}
	\label{1}
\end{figure}
\begin{figure}[h]
	\centering
	\resizebox{50pt}{!} {
		\Tree[.1/4 [t ][ [x ][t ]] ]
	}\ \ \ \ \ $\Rightarrow$\ \ \ \ \
	\resizebox{30pt}{!} {
		\Tree[.1/4 [r ][x ] ]
	}\\
\includegraphics[height=10mm, width=25mm]{
z_images/4_experimentations/2_reecriture_guidee/simplification_2.png}\ \ \ \ \ 
$\Rightarrow$\ \ \ \ \
\includegraphics[height=10mm, width=20mm]{
z_images/4_experimentations/2_reecriture_guidee/simplification_3.png}
	\caption{}
	\label{2}
\end{figure}
\begin{figure}[h]
	\centering
	\resizebox{70pt}{!} {
		\Tree[.1/4 [x ][t ][x ][x ]]
	}\ \ \ \ \ $\Rightarrow$\ \ \ \ \
	\resizebox{50pt}{!} {
		\Tree[.1/4 [x ][ [x ][x ]]]
	}\\
\includegraphics[height=10mm, width=25mm]{
z_images/4_experimentations/2_reecriture_guidee/simplification_4.png}\ \ \ \ \ 
$\Rightarrow$\ \ \ \ \
\includegraphics[height=10mm, width=20mm]{
z_images/4_experimentations/2_reecriture_guidee/simplification_5.png}
	\caption{}
	\label{3}
\end{figure}\newpage
\begin{figure}[h]
	\centering
	\resizebox{70pt}{!} {
		\Tree[.1/4 [t ][x ][x ][t ] ]
	}\ \ \ \ \ $\Rightarrow$\ \ \ \ \
	\resizebox{50pt}{!} {
		\Tree[.1/4 [ [r ][x ]][x ] ]
	}\\
\includegraphics[height=10mm, width=25mm]{
z_images/4_experimentations/2_reecriture_guidee/simplification_8.png}\ \ \ \ \ 
$\Rightarrow$\ \ \ \ \
\includegraphics[height=10mm, width=25mm]{
z_images/4_experimentations/2_reecriture_guidee/simplification_9.png}
	\caption{}
	\label{4}
\end{figure}
%\newpage
Ces règles ont été tirées de l’ensemble des arbres de la forme rythmique. Les
arbres manquants seront mis en annexe.

Les règles remplacent par un silence les continuations (t) qui sont au début
d’un temps. Cela est valable pour cette forme rythmique mais lorsqu’il y a des
ouvertures de charley, cela n’est pas toujours applicable.

\subsection*{Conclusion sur cette réécriture guidée}
La méthode des formes rythmiques étant basée sur une approche dictionnaire,
Le premier objectif de cette réécriture guidée est d’orienter la recherche
d’autres formes rythmiques par observation du jeu de données et de montrer
comment les construire pour agrandir la base de connaissance de Qparse pour la
transcription de la batterie.

\section{BILAN : résultats — évaluation — discussion}

Cette section regroupe les avancées qui ont été réalisées par rapport aux
objectifs de départ ainsi qu’une réflexion sur le moyen d’évaluer les résultats
de l’ADT avec Qparse. Nous avons améliorer le système de quantification de
Qparse pour la batterie, notamment le passage à la polyphonie avec les Jams. 

Nous avons pu obtenir des arbres de parsing corrects en améliorant les
grammaires avec des fichiers MIDI courts. 

Puis, une sortie MEI a aussi été obtenu (encore à vérifier).



Dans cette section, nous discuterons sur la pertinence de l’ensemble des choix
qui ont été faits. Nous ferons un bilan des différentes avancés qui ont été
faites ou non et nous tenterons d’en expliquer la ou les raisons.
\begin{itemize}
    \item Le choix de travailler avec lilypond et non verovio. Ce choix était
        motivé par la liberté totale concernant la notation de la batterie dont
        un et la disponibilité d’un set de notation de type agostini. C’est la
        seule application qui me permettait d’écrire la notation de la batterie
        exactement comme je le souhaitais.
    \item Avancé de la chaîne de traitement (nous sommes arrivé au arbres de
        parsing, nous avons traité le polyphonique (identification des
        regroupements de notes\footnote{fla ou accords entre autres…})
        $\Rightarrow$ Quelques arbres ont été obtenus sur des exemples simples
        (\footnote{exemple de 2 mesures, voir …})
    \item 2 dimensions de le travail fourni :\\
        - La volonté de pousser un exemple simple jusqu’au bout de la chaîne
        pour obtenir des résultats et une évaluation sur au moins un exemple ;
        - La réalité du travail à fournir pour faire avancer sur la chaîne de
        traitement.
        $\Rightarrow$ Une solution aurait été de considérer les arbres de
        parsing obtenus après le traitement du polyphonique comme un résultat
        local possible à évaluer au lieu d’attendre que la chaîne arrive
        jusqu’à la génération d’une partition mais cela n’était pas prioritaire
        pendant le stage.
    \item Création d’un jeu de forme rythmique basique réprésentatif des
        différents styles à recouvrir. Ce jeu n’a pas pu être créé, car comme
        vu plus haut, je me suis focalisé sur un exemple pour pouvoir le
        vérifier entièrement et dans l’espoir de pouvoir le tester en fin de
        chaîne.
\textbf{Évaluation}
     Matcher les motifs aurait été indispensable pour obtenir une quantité
        de résultats qui justifieraient une évaluation automatique permettant
        de faire des graphiques.\\
	 L’évaluation fut entièrement manuelle car :\\
	$\Rightarrow$ Très dure automatiquement : il faut comparer 2 partitions
    (réf \tab VS output)
Pour l’évaluation, il aurait fallu produire un module.\\
<dam>je ne sais pas si tu auras encore le temps de faire ça, sinon il faudra
décrire comment tu aurais aimé évaluer, proprement et sans résultats
chiffrés</dam>
L’évaluation est-elle automatique ou manuelle ?\\
Possibilité d’un export lilypond en arbre pour comparer l’ouput avec la
transcription manuelle.\\
Possibilité de transformer lilypond(output) et lilypond(ref) en ScoreModel ou
MEI pour les comparer et faire des statistiques. Si transformés en MEI :
diffscore de Francesco.
Possibilité de transformer lilypond(output) et lilypond(ref) en MusicXML pour
les comparer ou dans Music21.
L’expérimentation peut-être considérer comme une évaluation manuelle ?
(magicien d’Oz)\\
Lilypond vers MIDI + ouput vers MIDI $\Rightarrow$ Comparaison des MIDI
dumpés.\\

%\item Le ternaire jazz (voir expérience 2)
%\item Reconnaissance d’un motif sur le MIDI\\
%Reconnaître un motif (forme rythmique) sur une mesure de l’input (un fichier
%midi
%représentant des données audios)\\
%$\Rightarrow$ Motif (forme rythmique) reconnu : true ou false\\
%Si true :\\
%- Choisir la grammaire correspondante ;\\
%- Parser le MIDI ;\\
%- Appliquer les règles de réécritures (Séparation des voix et
%simplification)
%\item Nous travaillerons aussi sur la détection de répétitions sur
%plusieurs mesures afin de pouvoir corriger des erreurs sur une des mesures
%qui aurait dû être identique aux autres mais qui présente des différences.
%\item dans quelle catégorie mettre le shuffle ?\\
%\item écrire des règles de réécriture spécifique aux charley avec un forme
% rythmique
%approprié.

\end{itemize}

La transcription automatique de la batterie est un sujet passionnant mais
difficile : Obtenir la totalité des éléments nécessaires pour le mémoire
nécessiterait plus de temps. Une base solide spécifique à la batterie a
néanmoins été générée. Elle sera un bon point de départ pour les travaux futurs
dont plusieurs propositions sont énoncés dans le présent document.
