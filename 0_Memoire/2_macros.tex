%%%%%%%%%%%%%%%%%%%%%%%%%%%%%%%%%%%%%%%%%%%%%%%%%%%%%%%
%% EN-TETES ET PIEDS DE PAGE
\let\footruleskip\undefined
\usepackage{fancyhdr}
\pagestyle{fancy} % Activer le style de pages personnalisé
\fancyhf{} % Remettre à zéro des en-tête et pied de page
\setlength{\headheight}{14pt} % Fixer la hauteur de l'espace réservé à l'en-tête du haut

%%% Pas de numéro de page sur la première page des chapitres
\makeatletter
\let\ps@plain=\ps@empty
\makeatother

%===================== Style 1 =================================================
\fancyhead[RO,LE]{
\thepage
}
\fancyhead[LO]{\scshape \nouppercase{\rightmark}}
\fancyhead[RE]{\scshape \nouppercase{\leftmark}} 
\renewcommand{\headrulewidth}{.4pt}
\fancyfoot{}

%================================== Style 2 ====================================
\let\origdoublepage\cleardoublepage
\newcommand{\clearemptydoublepage}{
  \clearpage
  {\pagestyle{empty}\origdoublepage}
}
\let\cleardoublepage\clearemptydoublepage
\FrenchFootnotes
\AddThinSpaceBeforeFootnotes

%%%%%%%%%%%%%%%%%%%%%%%%%%%%%%%%%%%%%%%%%%%%%%%%%%%%%%%

\newcommand*\chapterstar[1]{
  \chapter*{#1}
  \addcontentsline{toc}{chapter}{#1}
  \markboth{#1}{#1}}


%%%%%%%%%%%%%%%%%%%%%%%%%%%%%%%%%%%%%%%%%%%%%%%%%%%%%%%
% ENVIRONNEMENTS DE THEOREMES
\theoremstyle{plain} % style plain
\newtheorem{theo}{Théorème}[chapter]
\newtheorem{cor}[theo]{Corollaire}
\newtheorem{prop}[theo]{Proposition}
\newtheorem{lem}[theo]{Lemme}
\newtheorem{conj}[theo]{Conjecture}
\newtheorem*{theoetoile}{Théorème} % théorème non numéroté
\newtheorem*{conjetoile}{Conjecture} % conjecture non numérotée

\theoremstyle{definition} % style definition
\newtheorem{defi}[theo]{Définition}
\newtheorem{exemple}[theo]{Exemple}
\newtheorem{question}[theo]{Question}
\newtheorem{remarque}[theo]{Remarque}
\newtheorem{notation}[theo]{Notation}

% Pour renommer ``preuve'' en ``démonstration''
\renewcommand{\proofname}{Démonstration}


%%%%%%%%%%%%%%%%%%%%%%%%%%%%%%%%%%%%%%%%%%%%%%%%%%%%%%%
% ENVIRONNEMENTS DEDICACE ET EPIGRAPHE
\newenvironment{dedicace}{%
  \newpage\thispagestyle{empty}e
  \hfill\begin{minipage}{100mm}\begin{flushright}\it}{%
  \end{flushright}\end{minipage}\vfill}

\newenvironment{epigraphe}{%
  \hfill\begin{minipage}{60mm}\begin{flushright}\footnotesize\it}{%
  \end{flushright}\end{minipage}\hspace*{7mm}\vfill}
