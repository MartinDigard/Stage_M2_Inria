\subsection{squant : parsing du fichier midi}
squant lit le midi\\
grammaire wta qui détermine le poid\\
La distance est automatiquement déterminée par squant\\
distance à l’input\\
complexité de la notation

On veut minimiser le coût et la distance $\Rightarrow$ Trouver un compromis.

./build/squant2 -h

Essayer le lire un fichier midi avec squant2
lire mesure par mesure
Regarder les wta(grammaire)

\subsubsection{Quelques tests de lecture midi}
\begin{verbatim}
Les 4 messages suivants sont présents dans tous les tests qui suivent :
[ info] schema file: test/schema/schema-01.wta (??? weight model option)
[warning] no declaration MAX\_GRACE in grammar file test/schema/schema-01.wta
[warning] no declaration TIMESIG in grammar file test/schema/schema-01.wta
[warning] MIDIfile has not joined tracks

./build/squant2 -v 4 -a test/schema/schema-01.wta -m 004_jazz-funk_116_beat_4-4.mid -config ./params.ini
[error] at least one of the options -bars or -barsec mandatory

./build/squant2 -verbosity 4 -schema test/schema/schema-01.wta -midi 004\_jazz-funk\_116\_beat\_4-4.mid -config ./params.ini -barsec 3

squant2: /home/martin/qparselib/src/schemata/SymbLabel.cpp:44: static label_t SymbLabel::make(unsigned char, SymbLabel::Kind, short unsigned int, short unsigned int): Assertion `info2 < 512' failed.
Abandon (core dumped)
\end{verbatim}

Tester squant2 avec le fichiers midi du corpus du gitlab\\\\

La commande suivante :
\begin{verbatim}
	build/squant2 -v 5 -a ./test/schema/schema-03-R.wta -m ~/corpus-master_qparselib/103-SaintSaens-elephant/perf/103_FJ.mid -config ./params.ini -mono -barsec 3.0 -ts 3/4	
\end{verbatim}
Donne :\\
%(1) 3(●, 2̅:2(●, ○), )\\
%(2) 3(●, 2̅:2(●, ●), )\\
%(3) 3(●, 2̅:2(●, ○), )\\
%(4) 3(2̅(2̅(2̅(●, ●), ⏑), ⏑), ⏑:2, )\\
%(5) 3(2̅(2̅(●, ○), ●), ●:2, )\\
%(6) 3(2̅(●, 2̅(●, 2̅(○, ●))), ⏑:2, )\\
%(7) 3(2̅(●, ●), ●:2, )\\
%(8) 3(2̅(●, 2̅(●, 2̅(⏑, ●))), 2̅(⏑, 2̅(●, ○)), ●)\\
%(9) 3(●, ●:2, )\\
%(10) 3(●, 2̅:2(●, 2̅(●, 2̅(⏑, ●))), )\\
%(11) 3(⏑, 2̅:2(●, ○), )\\
%(12) 3(●, ⏑, ●)\\
%(13) 3(●, 2̅:2(2̅(●, 2̅(○, ●)), 2̅(⏑, 2̅(○, ●))), )\\
%(14) 3(⏑, 2̅:2(●, 2̅(●, ○)), )\\
%(15) 3(●, ●:2, )\\
%(16) 3(●, ○:2, )\\

Pour comprendre les grammaires :\\
Regarder les fichiers wta commentés.\\
https://qparse.gitlabpages.inria.fr/docs/scientific/\\
A\_Parse-based\_Framework\_for\_Coupled\_RhythmQuantization\_and\_Score\_Structuring.pdf
Réfléchir au coût de notation (grace notes, etc.)
\newpage