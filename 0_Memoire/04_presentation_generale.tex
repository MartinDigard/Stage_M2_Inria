\subsection*{Le contexte}
Ce mémoire de recherche, effectué en parallèle d’un stage à l’Inria dans le cadre du master de traitement automatique des langues de l’Inalco, contient une proposition d’amélioration d’une chaîne de traitement d’ADT de bout en bout.\\
Même si ce sujet ne traite pas directement de langues naturelles, son objet est l’écriture automatique de partitions de musique à partir de données audios. Il nécessite donc la manipulation d’un langage musical codifié avec une grammaire (solfège, durées, nuances, volumes) et soulève de nombreuses problématiques proches de la reconnaissance de la parole.\\\\
\textit{Notes discussion Damien} :\\
\textit{\textbf{Musique} $\Rightarrow$ langue ou langage ? ;\\
\textbf{Partition musicale} $\Rightarrow$ Manière d’écrire la musique…}\\\\
La chaîne de traitement peut-être séparée en deux grandes parties :
\begin{itemize}
	\item Le traitement du signal à partir d’enregistrements audios de performances de batteurs et l’écriture des données sur des fichiers MIDI.
	\item La transformation des fichiers MIDI en partition de batterie (chaque fichier MIDI étant une partition)\\
\end{itemize}

Le sujet du stage se concentre sur la deuxième partie de cette chaîne et le sujet du mémoire est une proposition d’amélioration sur cette partie de la chaîne.

\subsection*{Sujet du stage}
Le but du stage est d’améliorer qparse, un outil de transcription et d’écriture automatique de la batterie (entre autre).\\
L'étude de modèles de langage (LM) incorporant certaines informations musicales de haut niveau nécessaires à la génération de partitions de qualité. On devrait en particulier considérer des hiérarchies d'événements de batterie induisant des placements temporels cohérents et se prêtant à des notations rythmiques faciles à lire pour un batteur entraîné ; voir \cite{foscarin:hal-01988990} pour des modèles structurés en arbre basés sur la théorie formelle du langage, que nous développons dans le contexte d'outils AMT plus généraux.
\subsection*{Ce mémoire}
Le mémoire propose de rechercher de rythmes génériques (\textit{motifs}) en amont dans la chaîne de traitement. Les \textit{motifs} sont prédéfinis avec des combinaisons possibles (\textit{gammes}) qui leur sont associées. Ces \textit{motifs} et leur \textit{gammes} respectives sont appelés \textit{systèmes}.\\L’usage des \textit{systèmes} a pour objectif de fixer des choix le plus tôt possible dans la chaîne de traitement afin de simplifier le reste des calculs en éliminant une partie d’entre eux. Ces choix concernent notamment la métrique, la séparation des voix ainsi que les règles de réécriture.
\subsection*{Problématique traitée}	
	L’écriture musicale offre de nombreuse possibilités d’écriture pour un rythme donné, certains de ces choix sont plus pertinents que d’autres en fonction du contexte.\\
	Reconnaître la métrique d’une partition, la façon de regrouper les notes par les ligatures (séparation des voix), ou simplement le choix pour les différentes continuations possibles (notes pointées, liaisons, silences, etc.) représentent autant de possibilités que de difficultés.
\subsection*{Plan du mémoire}


Introduction générale\\\\
Partie I : Contexte général
\begin{itemize}
	\item État de l’art
	\item La transcription automatique
	\item Les méthodes\\
\end{itemize}
Partie II : Expérimentations
\begin{itemize}
	\item Corpus
	\item Résultats
	\item Discussion\\
\end{itemize}
Conclusion générale\\