\section*{Introduction}
Dans ce chapitre, nous détaillerons les méthodes que nous avons employées pour
la TAB. Nous commencerons par donner une description de la notation de la
batterie ainsi que des précisions sur lilypond, l’outil utilisé pour les
transcriptions manuelles et les raisons qui ont motivé le choix de cet outil.
Nous présenterons ensuite une modélisation de notation de la batterie pour sa
représentation en arbres syntaxiques suivie d’une présentation de qparse
\footnote{\url{https://qparse.gitlabpages.inria.fr/}}, un outil de transcription qui
est développé à l’Inria, l’Université de Nagoya et plusieurs développeurs au
sein du laboratoire Cedric au CNAM. Enfin, nous présenterons les
formes rythmiques, une représentation théorique qui
permet, par le biais de \textit{patterns} accompagnés de leur combinaisons
spécifiques associées, de détecter les signatures rythmiques de performances
non-quantifiées et de restreindre les choix de règles à appliquer afin de 
simplifier les calculs de réécriture.

\section{La notation de la batterie}
\label{notation_batterie}
Pour la transcription, j’ai choisi d’utiliser une notation inspirée du recueil
de pièces pour batterie de J.-F. Juskowiak \cite{jusko} et des méthodes de
batterie Dante Agostini \cite{ago_meth_3}, car je trouve la position des éléments
cohérente et intuitive.

%\subsection*{Les hauteurs et les têtes de notes}
%\label{hauteurs}
\begin{figure}[h]
\centering
\includegraphics[height=69mm, width=115mm]{
z_images/3_methodes/0_notation_de_la_batterie/batterie.png}
\caption{Les instruments de la batterie}
\label{instru_batt}
\end{figure}

\begin{figure}[!h]
\centering
\includegraphics[height=25mm, width=130mm]{
z_images/3_methodes/0_notation_de_la_batterie/2_hauteurs_et_tete_de_notes.png}
\caption{Les hauteurs et têtes de notes}
\label{haut}
\end{figure}

\begin{table}[h]
\centering
\begin{tabular}{|c|c|c|} \hline
Noms figure \ref{instru_batt} & codes figure \ref{haut}  & référence \\ \hline
Pédale de charleston & pf ou po & charley fermé ou ouvert au pied \\
Grosse caisse & gc & grosse caisse \\
Tom basse & tb & tom basse \\
Caisse claire & cc & caisse claire \\
Tom médium & tm & tom médium \\
Tom alto & ta & tom alto \\
Cymbales charleston & cf ou co & charley fermé ou ouvert à la main \\
Cymbale ride & rd & ride \\
Cymbale crash & cr & crash \\ \hline
	\end{tabular}
	\caption{Les noms des instruments de la batterie}
	\label{nom_instru_batt}
\end{table}\newpage
La figure \ref{instru_batt}\footnote{Les noms des instruments ont été changés
    sur cette image qui provient du site \url{
https://www.superprof.fr/blog/composition-instrument-percussion/}} montre une
batterie standard avec tous les instruments habituellement présents sur une
batterie et la figure \ref{haut} donne leur représentation sur une partition.\\

Le tableau \ref{nom_instru_batt} donne dans l’ordre~ :
\begin{enumerate}
    \item les noms des instruments sur la figure \ref{instru_batt}~ ;
    \item leurs codes respectifs dans la figure \ref{haut}~ ;
    \item les noms que j’utiliserai dans le présent document pour y référer.
\end{enumerate}
Les figures \ref{instru_batt}, \ref{haut} et le tableau \ref{nom_instru_batt}
peuvent aider à comprendre pourquoi je trouve la notation des méthodes Agostini
cohérente et intuitive. En effet, les hauteurs sur la portée représentent~ :
\begin{enumerate}
	\item La hauteur physique des instruments~ :\\
	La caisse claire est centrale sur la portée et sur la batterie (au niveau
    de la ceinture, elle conditionne l’écart entre les pédales et aussi la
    position de tous les instruments basiques d’une batterie).\\
	Tout ce qui est en-dessous de la caisse claire sur la portée est en dessous de
    la caisse claire sur la batterie (pédales, tom basse)~ ;\\
	Tout ce qui est au-dessus de la caisse claire sur la portée, l’est aussi
    sur la batterie.\\
	\item La hauteur des instruments en terme de fréquences~ :\\
	Sauf pour le charley au pied et si on les sépare en trois groupes
    (grosse caisse, toms et cymbales), de bas en haut, les instruments vont du
    plus grave au plus aigu.
\end{enumerate}

\subsection*{Les durées}
\label{hho}
Comme nous venons de le voir sur la figure \ref{haut}, la majorité des
instruments de la batterie sont représentés par les têtes des notes. De plus, le seul instrument dont le son
peut être arrêté de manière quantifiée et dont la durée sonore nous intéresse
est le charley\footnote{Je ne prendrais pas en compte l’arrêt des cymbales à la
main car ce phénomène n’existe pas dans les fichiers MIDI.}.
%\florent{certaines têtes de notes vides alors que leur durée n’est pas celle
%des blanches? expliquer les différences avec la notation conventionnelle cf
%1.4}

Par conséquent~ :
\begin{enumerate}
    \item les durées — sauf pour le charley — représenterons un écart temporel
        entre les notes et non une durée sonore et elles pourront donc être
        rallongées à l’aide de silences~ ;
    \item les symboles rythmiques concernant les têtes de note ne pourront pas
        être utilisés pour exprimer les durées. Cela est valable aussi pour la
        présence ou non de la hampe puisque ce phénomène n’existe qu’avec les
        têtes de notes de type cercle-vide (opposition blanche-ronde). L’usage
        des blanches existe dans certaines partitions de batterie
        \cite{system_drums} mais cela reste dans des cas très rares. Certains
        logiciels permettent de faire des blanches avec des symboles
        spécifiques à la batterie ou aux percussions mais leur lecture reste
        peu aisée et leur utilisation pour la batterie est rarissime.\\
\end{enumerate}

En résumé~ :
\begin{itemize}
    \item toutes les notes ont une hampe~ ;
    \item une notes dont la hampe n’a pas de crochet est toujours une noire~ ;
    \item à part pour le charley ouvert, les durées n’expriment pas la durée
        d’un son mais une distance temporelle entre deux notes.
    \item à part pour le charley ouvert, la durée d’une note peut être
        prolongée par un silence (exemple~ : une noire + un soupir pour exprimer
        une blanche)\\
\end{itemize}
La durée d’une note peut être prolongée par divers symboles~ :
\begin{itemize}
	\item le point~ : il rallonge la durée d’une note de la moitié de sa valeur~ ;
        dans l’exemple 3 de la figure \ref{point_liaison}, la deuxième note
        est une noire pointée, elle vaut donc la durée d’une noire + une croche
        (ou de trois croches)~ ;
	\item la liaison~ : elle rallonge la durée de la première note de la durée
        de la deuxième. La deuxième note de l’exemple 4 de la figure
        \ref{point_liaison} est une croche qui est liée à une noire, sa durée
        est donc équivalente à celle d’une croche + une noire (ou de trois
        croches)~ ;
    \item les silences (sauf pour les ouvertures de charley).\\
\end{itemize}

\begin{figure}[h]
	\centering
	\includegraphics[height=80mm, width=40mm]{
    z_images/3_methodes/0_notation_de_la_batterie/3_point_et_liaison.png}
	\caption{Le point et la liaison}
	\label{point_liaison}
\end{figure}

Un autre élément concernant la notation des durées en batterie est la nécessité
de faire ressortir la pulsation\footnote{La position des temps}, de la rendre
visuelle. La première chose à prendre en compte pour analyser la figure
\ref{point_liaison} est donc la nécessité de regrouper les notes par temps à
l’aide des ligatures. Le deuxième point est de s’arranger pour qu’il y ait une
indication visuelle au début de chaque temps.

\begin{itemize}
    \item Exemple 1~ : l’ouverture de charley est quantifiée mais les notes ne
        sont pas regroupées par temps.
    \item Exemple 2~ : ici, la liaison permet de regrouper les notes par temps
        en obtenant le même rythme que dans l’exemple 1.
    \item Exemple 3 et exemple 4~ : les deux exemples sont valables mais le
        deuxième est le plus souvent utilisé car la liaison donne un repaire
        visuel sur le temps.\\
\end{itemize}

En cas de nécessité de prolonger la durée d’une note au-delà 
de son temps de départ (syncope) et si cette note ne correspond pas à une
ouverture de charley, elle sera prolongée sur le temps suivant à l’aide de
silences dont le premier sera positionné sur le temps. Si la note syncopée est
une ouverture de charley, on privilégiera la liaison pour sa prolongation (exemple 4).

\subsection*{Les silences}
Les silences sont parfois utilisés pour noter les fermetures de charley (après
une ouverture). Les fermetures du charley sont notées soit par un silence
(correspondant à une fermeture de la pédale), soit par un écrasement de
l’ouverture par un autre coup de charley fermé, au pied ou à la main.

\begin{figure}[h]
	\centering
	\includegraphics[height=40mm, width=40mm]{
    z_images/3_methodes/0_notation_de_la_batterie/5_silence_joue.png}
	\caption{Les silences en batterie}
	\label{silence joue}
\end{figure}

L’écriture littérale de contenu MIDI peut ressembler à l’exemple 1 de la figure
\ref{silence joue}. Sur cet exemple, le son de l’ouverture de charley est
arrêté par une pression du pied sur la pédale et c’est ce que le batteur joue
dans les faits. Mais il apparaît intuitivement que le but de la première note
du deuxième temps n’est pas de générer un son de charley au pied mais
uniquement de stopper l’ouverture. La notation de l’exemple 2 de la figure
\ref{silence joue} serait donc préférable car elle représente mieux l’intention
de ce rythme et elle n’empiète pas sur une potentielle voix basse qui pourrait
le compléter (on évite une écriture surchargée).

Lorsqu’une note est un charley ouvert, il faudra donc prendre en compte la note
suivante pour l’écriture~ :
\begin{enumerate}
    \item si c’est un charley fermé joué à la main $\Rightarrow$ la note sera
        un charley fermé joué à la main (cf)~ ;
    \item si c’est un charley fermé joué au pied $\Rightarrow$ la note sera un
        silence.
\end{enumerate}
%La deuxième règle sera soumise au cadre imposé par certaines
%formes rythmiques pour lesquelles le charley joué au pied devra rester
%tel quel. 

\subsection*{Les équivalences rythmiques}
Pour les instruments mélodiques, dans le cas de notes dont la durée de l’une à
l’autre est ininterrompue et si leur durée initiale est prolongée, seuls la
liaison et le point permettent des notations équivalentes. Mais pour la
batterie et à part dans le cas des ouvertures de charley (voir section
\ref{hho}), seules comptent des dates de début (\textit{onsets})~ : la durée du son n’a
pas d’importance. L’usage des silences pour combler la distance rythmique entre
deux notes devient donc possible.

Cela pris en compte, et étant donné que les indications de durée dans les têtes
de notes sont peu recommandées (voir section \ref{hho}), l’écriture à l’aide de
silences sera privilégiée comme indication de durée sauf dans les cas où cela
reste impossible. Ce choix à pour but de n’avoir qu’une manière d’écrire toutes
les notes, quelles que soient leur tête de note (sauf pour le charley).

\begin{figure}[h]
	\centering
	\includegraphics[height=20mm, width=75mm]{
    z_images/3_methodes/0_notation_de_la_batterie/6_equivalence.png}
	\caption{Les équivalences rythmiques}
	\label{equivalence}
\end{figure}

Sur la figure \ref{equivalence}, théoriquement, il faudra choisir la notation
de la deuxième mesure mais dans certains contextes, pour des raisons de
lisibilité ou de surcharge, la version sans les silences de la troisième mesure
pourra être choisie.

\subsection*{Les voix}


En batterie, une voix est théoriquement l’ensemble des instruments qui, à eux
seuls, constituent une phrase rythmique. Mais en pratique, les instruments
peuvent aussi être divisés par voix dans le but de ne pas surcharger la
notation ou pour que leur disposition soit représentée sur la
partition (voir section \ref{notation_batterie}).
Les voix sont caractérisées par l’orientation des hampes et plus précisément
par les ligatures si les hampes sont dans la même direction (voir figure
\ref{afro_latin}).

\begin{figure}[h]
	\centering
	\includegraphics[height=65mm, width=60mm]{
    z_images/3_methodes/0_notation_de_la_batterie/7_voix.png}
	\caption{La séparation des voix}
	\label{sep_voix}
\end{figure}
Sur la figure \ref{sep_voix}, il faudra faire un choix entre les exemples 1, 2
et 3 qui sont trois façons équivalentes d’écrire le même rythme.
Ce choix se fera en fonction des instruments joués, de la nature plus ou moins
systématique de leurs phrasés, et des associations logiques entre les
instruments dans la distribution des rythmes sur la batterie (voir la section
\ref{sys_sep_voix}).

\subsection*{Les accentuations et les ghost notes}
«~Certaines notes dans une phrase musicale doivent, ainsi que les différentes
syllabes d’un mot, être accentuées avec plus ou moins de force, porter une
inflexion particulière.~» \cite{danhauser}
\begin{figure}[h]
\centering
\includegraphics[height=25mm, width=75mm]{
z_images/3_methodes/0_notation_de_la_batterie/8_accents_et_ghost-notes_0.png}
\caption{Les accents et les ghost notes}
\label{accents_et_gn}
\end{figure}

Théoriquement, tous les instruments peuvent être accentués (voir la section
\ref{velocite}), mais la figure \ref{accents_et_gn} représente ceux dont les
accents sont presque toujours bien articulés dans le jeu standard des batteurs.
En outre, les instruments qui ne sont pas représentés sur cette
figure ne sont presque jamais accentués dans les partitions et ne sont pas
présents de manière significative dans le jeu de données (voir section \ref{gmd}) utilisé dans ce
travail.

Les accents sont marqués par le symbole «~>~». Ils sont positionnés au-dessus
des notes représentant des cymbales et en-dessous des notes représentant des
toms ou la caisse claire. Ce choix a été fait pour la partition de la figure
\ref{partition_ref} car elle est plus lisible ainsi, mais ces choix devront
être adaptés en fonction des différentes formes rythmiques reconnues
(voir la section \ref{systemes_methodes}). Par exemple, pour les
formes rythmiques jazz, les ligatures pour les toms et la caisse
claire seront dirigées vers le bas, il faudra donc mettre les symboles
d’accentuation correspondants au-dessus des têtes de notes.

La dernière note de la figure \ref{accents_et_gn} montre un exemple de notation
pour une ghost note jouée à la caisse claire. Une ghost note
\cite{lexique_drum} est une note de faible volume sonore mais jouée fermement.
Les ghost notes servent le plus souvent à donner le débit d’un rythme (ses
subdivisions) pour le rendre plus dansant (lui donner plus de «~groove~» ou de
«~swing~»). Le parenthésage a été choisi car il peut être utilisé sur n’importe
quelle note sans changer la tête de note.

Toutes les notes de la figure \ref{accents_et_gn} sont exposées en situation
réelle dans la figure \ref{exemple_acc_et_gn}. 
\begin{figure}[h]
\centering
\includegraphics[height=20mm, width=110mm]{
z_images/3_methodes/0_notation_de_la_batterie/8_accents_et_ghost-notes_1.png}
\caption{Exemple pour les accentuations et les ghost notes}
\label{exemple_acc_et_gn}
\end{figure}

\subsection*{Les flas}
Le fla est une appogiature qui consiste à jouer deux coups presque simultanés dont
le premier est une ghost note et le deuxième une note normale ou accentuée.
\begin{figure}[h]
    \centering
    \includegraphics[height=10mm, width=8mm]{
    z_images/3_methodes/0_notation_de_la_batterie/fla_def.png}
    \caption{La notation du fla}
\end{figure}

\section{La transcription manuelle avec lilypond}
\label{tm}
Mis à part les figures du chapitre 1 et certains exemples d’analyses de la
section \ref{analyses_et_TM}, toutes les partitions et figures de ce document
ont été généré avec lilypond\footnote{\url{http://lilypond.org/index.fr.html}}.\\

\subsection*{Présentation de lilypond}
«~LilyPond est un logiciel de gravure musicale, destiné à produire des
partitions de qualité optimale. Ce projet apporte à l’édition musicale
informatisée l’esthétique typographique de la gravure traditionnelle. LilyPond
est un logiciel libre rattaché au projet GNU.~» \footnote{Page d’accueil du site \url{http://lilypond.org/index.fr.html}}\\

En raison de la grande liberté de choix que permet lilypond et du fait qu’une
configuration pour la notation de type Agostini est disponible, je considère
que lilypond est actuellement le meilleur choix pour transcrire de la batterie.\\

\begin{figure}[h]
    \centering
    \includegraphics[height=40mm, width=30mm]{
    z_images/3_methodes/transcription_manuelle/drum_perso_1}
    \includegraphics[height=70mm, width=50mm]{
    z_images/3_methodes/transcription_manuelle/extrait_code.png}
    \caption{Extraits de code lilypond}
    \label{extrait_code}
\end{figure}

Sur la figure \ref{extrait_code}~ :
\begin{itemize}
    \item à gauche, une configuration aménagée pour la notation de type Agostini.
    \item à droite, le début de code mesure par mesure pour la voix haute
        d’une partition (la première ligne sert à prendre en compte le fichier
        de gauche).
\end{itemize}

\begin{figure}[h]
    \includegraphics[height=120mm, width=160mm]{
    z_images/4_experimentations/1_analyses/3_partition.png}
    \caption{Transcription de partition avec lilypond}
	\label{partition_ref}
\end{figure}

La partition de la figure \ref{partition_ref} est le résultat du code de la
figure \ref{extrait_code} (la totalité du code est sur
\url{https://github.com/MartinDigard/Stage_M2_Inria}). Cette partition a été totalement transcrite
manuellement avec lilypond par analyse des fichiers MIDI et audio
correspondants.

\begin{itemize}
    \item Difficultés principales~ : trouver une application permettant de
        choisir librement la notation de la batterie. Lylipond le permet mais
        beaucoup de recherches ont été nécessaires pour comprendre l’ensemble
        des fonctionnalités permettant de faire fonctionner la notation
        «~Agostinienne~» ainsi que les diverses subtilités de notations
        (accents, ghost notes, flas, …).\\
        lylipond reste néanmoins un choix très agréable, une fois ces
        difficultés surmontées.
    \item Écrire la partition de la figure \ref{partition_ref} m’a pris
        beaucoup de temps car j’ai dû chercher comment écrire chaque nouvel
        évènement, mais les autres transcriptions ont été beaucoup plus rapide
        et très aisées.
    \item Même si cela représente un investissement au départ, je recommande
        lylipond pour écrire la batterie et je pense que c’est meilleur outil
        pour cette tâche pour le moment.
    \item Dans les autres logiciel d’édition de type musescore, la batterie
        est toujours confiné au système de notation américain.
    \item Pour une comparaison entre système de notation américain et le système de notation Agostini,
        voir section \ref{flas} est comparer les notations TM (Agostini) et
        TA (américain).
\end{itemize}


\section{Modélisation pour la transcription}
\label{modelisation_transcription}
\subsection*{Les pitchs}
\begin{table}[h]
	\centering
	\begin{tabular}{|c|c|c|} \hline
		Codes & Instruments & Pitchs \\ \hline
		cf & charley-main-fermé & 22, 42 \\
		co & charley-main-ouvert & 26 \\
		pf & charley-pied-fermé & 44 \\
		rd & ride & 51 \\
		rb & ride-cloche (bell) & 53 \\
		rc & ride-crash & 59 \\
		cr & crash & 55 \\
		cc & caisse claire & 38, 40 \\
		cs & cross-stick & 37 \\
		ta & tom-alto & 48, 50 \\
		tm & tom-medium & 45, 47 \\
		tb & tom-basse & 43, 58 \\
		gc & grosse caisse & 36 \\ \hline
	\end{tabular}
	\caption{Les codes, l’identités et les pitchs des instruments}
	\label{pitchs_instru}
\end{table}
Le tableau \ref{pitchs_instru} présente dans l’ordre, les codes des
instruments, leur identité (instrument ou partie d’un instrument, joué avec les
mains ou avec les pieds), le ou les pitchs qui lui sont associés.

Plusieurs pitchs peuvent parfois désigner le même instrument afin de pouvoir
supporter des kits \footnote{Les batteries électroniques permettent de choisir
les sons que l’on veut donner à chaque instruments parmi des kits prédéfinis
adaptés à différents style.} de batterie plus larges (avec par exemple plusieurs
toms basses qui n’auraient pas tous exactement la même sonorité) ou simplement
de styles différents (pour chaque kit standard, ce sont les mêmes intruments
mais de styles différents)\footnote{Par exemple, les peaux des toms jazz
raisonnent alors que celles des toms rock sont mates.}. J’ai regroupé les pitchs des
différents types d’un même instrument dans une seule ligne du tableau portant
le nom du type de cet instrument. Ainsi,
plusieurs toms basses différents dans les données MIDI deviennent tous un tom
basse d’une batterie standard et la partition finale pourra être jouée sur
n’importe quel kit de batterie standard.

Malgré le large panel de pitchs disponibles, il semblerait qu’aucun pitch ne
désigne le charley ouvert joué au pied («~po~» de la figure \ref{haut}).
Pourtant, dans la batterie moderne, plusieurs rythmes ne peuvent fournir le son
du charley ouvert qu’avec le pied car les mains jouent autre chose en même
temps. Cela doit en partie être dû à l’utilisation des boîtes à rythmes
en musique assistée par ordinateur qui ne nécessitent pas de faire des choix conditionnés par les
limitations humaines (2 pieds, 2 mains, et beaucoup plus d’instruments…)

\subsection*{La vélocité} \label{velocite}
La vélocité déterminera si les notes sont accentuées ou sont des ghost notes.
Pour les codes, je propose d’ajouter un suffix («~a~» pour accent et «~g~» pour
ghost note) à la fin du code d’une note accentuée ou d’une ghost note.
Les choix pour déterminer si les notes sont accentuées ou sont des ghost notes
seront donnés dans la section \ref{partition_entiere}.

\subsection*{Les arbres de rythmes}
Les arbres de rythmes représentent un rythme dont les possibilités de notation
sur une partition sont théoriquement multiples. Les branchements sont des
divisions d’intervalles temporels et les feuilles sont des évènements musicaux
commençant au début de l’intervalle \cite{Laurson1996PatchWorkA}
\cite{Bresson_openmusicvisual} .\\

La figure \ref{ex_arbre_1} est une représentation qui fonctionne avec les 3 exemples de la figure
\ref{sep_voix} en arbre de rythmes avec les codes de chaque instrument~ :\\

\begin{figure}[h]
	\Tree[ [ [rd\\gc ][ [rd\\pf ][rd ]]]
	[ [rd\\cc ][ [rd\\pf ][rd ]]]
	[ [rd\\gc ][ [rd\\pf ][rd ]]]
	[ [rd\\cc ][ [rd\\pf ][rd ]]] ]
    \caption{Exemple d’un arbre de rythmes avec les codes des instruments}
    \label{ex_arbre_1}
\end{figure}

La figure \ref{ex_arbre_2} montre le même arbre dont les codes des instruments sont remplacés par
leurs données MIDI respectives~ :\\

\begin{figure}[h]
	\Tree[ [ [51\\36 ][ [51\\44 ][51 ]]]
	[ [51\\38 ][ [51\\44 ][51 ]]]
	[ [51\\36 ][ [51\\44 ][51 ]]]
	[ [51\\38 ][ [51\\44 ][51 ]]] ]
    \caption{Exemple d’un arbre de rythmes avec les pitchs des instruments}
    \label{ex_arbre_2}
\end{figure}

Chacun des trois exemples de la figure \ref{sep_voix} sont représentés par chacun des
deux arbres syntaxiques ci-dessus. On voit bien ici l’avantage de cette représentation
pure des rythmes car elle permet de tester plusieurs notations équivalentes pour un même rythme.

%<dam>complète un peu en précisant qu’on voit bien ici l’avantage des arbres
%pour analyser ou construire la structure (les phrases ?) musicale</dam>

\section{Analyse syntaxique pour la transcription}

\begin{figure}[h]
\centering
\includegraphics[height=80mm, width=125mm]{
z_images/3_methodes/1_Analyse_syntaxique/schema_qparse.png}
\caption{Présentation de qparse}
\label{presentation_qparse}
\end{figure}

Comme le montre la figure \ref{presentation_qparse}, qparse\footnote{
\url{https://qparse.gitlabpages.inria.fr}} est un outil pour la transcription
musicale qui produit une partition structurée à partir d’une performance symbolique séquentielle et non
quantifiée, . Il effectue conjointement des
tâches de quantification rhythmique et d’inférence de la structure de la
partition à l’aide de techniques d’analyse syntaxique (\textit{parsing}). Le
but du \textit{parsing} est en effet la structuration d’une représentation
séquentielle en entrée (un mot fini), suivant un modèle de langage
\cite{grune2007parsing}.

Dans le cas de qparse, le "mot" d’entrée est typiquement au format MIDI, et le
modèle de langage est une grammaire d’arbres pondérés représentant des
préférences en terme de notation musicale à produire \cite{droste2009handbook}.
basée sur des algorithmes d’analyse syntaxique pour les grammaires
arborescentes pondérées. En prenant en entrée une performance musicale
symbolique (séquence de notes avec dates et durées en temps réel, typiquement
un fichier MIDI), et une grammaire hors-contexte pondérée décrivant un langage
de rythmes préférés, il produit une partition musicale. Plusieurs formats de
sortie sont possibles, dont les formats XML (MEI, MusicXML,…) ou lilypond.\\

Les principaux contributeurs sont~ :
\begin{itemize}
	\item Florent Jacquemard (Inria)~ : développeur principal.
	\item Francesco Foscarin (PhD, CNAM)~ : apprentissage~ ; Evaluation.
	\item Clement Poncelet (Salzburg U.)~ : integration de la librairie Midifile
        pour les input MIDI.
	\item Philippe Rigaux (CNAM)~ : production de partition au format MEI et de
        modèle intermédiaire de partition en sortie.
	\item Masahiko Sakai (Nagoya U.)~ : mesure de la distance input/output pour
        la quantification et CMake framework~ ; évaluation.
\end{itemize}

\subsection*{Les enjeux}
Un des problèmes de qparse était qu’il soit limité au monophonique. Il était
donc impossible de traiter plusieurs voix et de reconnaître les accords. Ce qui
est problématique pour la batterie étant donné que c’est un instrument dont l’essence
est la coordination de plusieurs sons à la fois. Ce problème a été en partie résolu
en regroupant les notes de faibles distances mutuelles dans des clusters appelés
«~Jam~» dont nous parlerons plus loin.
Un autre problème du MIDI avec qparse est le fait d’avoir deux symboles en entrée
pour un seul généré en sortie. Ce problème est moins génant pour la batterie car
nous avons pu ignorer tous \textit{offsets}.
La quantification nécessite d’ajuster les dates des notes une fois leur emplacement
déduit tout en minimisant la distance entre le midi et la représentation en arbre. La grammaire pondérée sert à gérer cet équilibre.
\subsection*{La grammaire}
\label{gram}
Il s’agit d’une grammaire hors-contexte pondérée dont la structure agit comme
un langage \textit{a priori} de notation rythmique. Les règles de la grammaire et
certaines valeurs de poids indiquent des rythmes et des écritures (pour les
ligatures notamment) à favoriser.\\

\begin{figure}[h]
    \centering
    \includegraphics[height=40mm, width=90mm]{
    z_images/4_experimentations/3_developpement/0_midi_2bars_fill.png}
    \caption{Fichier MIDI pour les tests de grammaires}
    \label{midi_gram}
\end{figure}

Les exemples suivants ont été écrits pour le fichier MIDI de la figure
\ref{midi_gram}. Ce fichier contient les informations de trois mesures dont les
deux premières contiennent des noires, des croches (et des doubles-croches pour
la deuxième) et dont la troisième est vide. La signature rythmique est
4/4.  

% framebox ? (suggestion Louis)
\begin{figure}[h]
    \centering
    \begin{verbatim}
        0 -> C0                1
        0 -> E1                1
        0 -> U4(1, 1, 1, 1)    1
        1 -> C0                1
        1 -> E1                1
        1 -> T2(2, 2)          1
        1 -> T4(4, 4, 4, 4)    1
        2 -> C0                1
        2 -> E1                1
        4 -> C0                1
        4 -> E1                1
        4 -> E2                1
        \end{verbatim}
        \caption{Exemple de grammaire}
        \label{ex_gram}
\end{figure}

Sur la figure \ref{ex_gram}, chaque ligne est une règle. La colonne de gauche
contient des symboles non-terminaux (0, 1, 2, 4). Cette grammaire sépare les temps par ligatures au niveau de la mesure. Puis
elle autorise, au niveau du temps, les divisions par deux (croches) et par
quatre (doubles-croches). Ces divisions seront reliées par des ligatures. Tous les poids sont réglés sur 1 à titre expérimental. L’arbre de parsing en
résultant (figure \ref{sortie_arbre}) est considéré comme «~convaincant~» car il découpe correctement les
mesures et les temps.
\begin{figure}[h]
    \centering
\resizebox{350pt}{!} {
\Tree[.Mesure\ 1
[.Temps\ 1 [0-ON\\1-ON\\2-ON ][3-OFF\\4-OFF\\5-OFF ][6-ON\\7-ON ][8-OFF\\9-OFF ]]
[.Temps\ 2 [10-ON\\11-ON ][12-OFF\\13-OFF ][14-ON\\15-ON ][16-OFF\\17-OFF ]]
[.Temps\ 3 [18-ON\\19-ON\\20-ON ][21-OFF\\22-OFF\\23-OFF ][24-ON\\25-ON ][26-OFF\\27-OFF ]]
[.Temps\ 4 [28-ON\\29-ON\\30-ON ][31-OFF\\32-OFF\\33-OFF ][34-ON\\35-ON ][36-OFF\\37-OFF ]]
]}
\caption{Arbre en sortie de qparse}
\label{sortie_arbre}
\end{figure}
%C signifie ??\\
%E signifie \textit{Event}\\
%U signifie \textit{Unbeamed} $\Rightarrow$ Au niveau de la mesure, quand il y
%a
%4 éléments de type 1, il ne doivent pas être liés par des ligatures.\\
%
%Dans la prochaine figure, nous sommes au niveau des temps (musical)
%
%Ici, on indique que s’il y a 2 éléments de type 2 (croches) ou 4 éléments de
%type 4 (doubles-croches), il doivent être reliés au par des ligatures.
%
%
%Les temps de la première mesure du fichier MIDI sont bien quantifié mais ceux
%de la deuxième mesure présentent quelques défauts de quantification visibles
%dès le premier temps.\\\\
%\resizebox{300pt}{!} {
%\Tree[.Mesure\ 2
%[.Temps\ 1 [38-ON ][ [39-OFF ][40-ON ] ][ [41-OFF\\42-ON ][43-ON ] ][
%[44-OFF\\45-OFF ][46-ON ] ]]
%]}\\\\\\
%Les Onsets sont correctement triés au niveau des doubles croches mais
%certaines doubles croches sont inutilement subdivisées en triples croches (les
%2ème, 3ème et 4ème doubles croches sur le premier temps ci-dessus).\\\\
%2ème exemple~ :}\\
%Après une augmentation du poids des triples croches dans la grammaire (monté
%de 1 à 5)et une baisse de tous les autres poids (descendu de 1 à 0.5), et mis
%à part le troisième temps de la 2ème mesure, tous les Onsets sont bien triés
%%et aucuns ne sont subdivisés.
%

\subsection*{Le parsing}
Les données MIDI sont quantifiées, les notes de dates proches sont alignées et
les relations entre les notes sont identifiées (accords, fla, etc…)~ ; un arbre
syntaxique global est créé (figure \ref{sortie_arbre}). 

\subsection*{La représentation intermédiaire}
\label{regles}
\begin{itemize}
    \item Les pitchs sont remplacés par les codes des instruments (voir tableau
    \ref{nom_instru_batt}~ ;
    \item réécriture 1~ :\\
    séparation des voix $\Rightarrow$ un arbre par voix
    $\Rightarrow$ représentation intermédiaire (RI)~ ;
    \item réécriture 2~ :\\
    simplification de l’écriture de chaque voix de la RI.\\
\end{itemize}
Cette procédure sera détaillée dans la section \ref{reecriture_guidee}.
\section{Les formes rythmiques}
\label{systemes_methodes}
Il existe en batterie des motifs rythmiques répétés (joués en boucle). Ces
motifs sont le résultat de la coordination de plusieurs instruments de la
batterie, je les nommerai motifs dans la suite du document. Très
souvent, un autre instrument est joué de manière indépendante sur le
motif mais en respectant la cohérence rythmique du motif. Je
nommerai gamme l’ensemble des combinaisons possibles pour cet autre
instrument. La gamme d’un motif est toujours relative à sa
signature rythmique. Enfin, j’ai nommé forme rythmique (FR)
le couple motif-gamme.

\subsection*{Objectifs}
Le but est d’avoir des schemas types (les formes rythmiques) pour
calculer la séparation en voix. Cela constituerait une heuristique pour éviter
d’avoir à explorer une grande combinatoire.

Un ensemble de formes rythmiques comprenant leur signatures rythmiques
respectives et leurs règles spécifiques de réécriture sera nécessaire.

Une fois un système défini, la transcription à partir de l’entrée MIDI sera
facilitée par la reconnaissance de la FR qui contraindra le processus, il
premettra de~ :
\begin{itemize}
	\item définir une signature rythmique~ ;
	\item choisir une grammaire appropriée~ ;
	\item fournir les règles de réécriture (séparation des voix et
        simplification.
\end{itemize}

\subsection*{Définitions}

\begin{itemize}
    \item motif~ : rythmes coordonnés joués avec deux ou trois
        instruments coordonnés en boucle (répartis sur 1 ou 2 voix)~ ;
    \item gamme~ : ensemble des combinaisons jouées par un autre
        instrument sur le motif (réparti sur 1 voix)~ ;
    \item forme rythmique~ : motif + gamme.\\
\end{itemize}

Les motifs sont fixes, contrairement aux gammes, qui regroupent
l’ensemble des possibilités pouvant être rencontrées en situation réelle
(sur une partition ou lors d’une performance de batterie).

Un motif détermine la signature rythmique d’une
forme rythmique , et donc de sa gamme. Il détermine aussi
avec quel instrument unique sera jouée la gamme et comment tous les
instruments de la forme rythmique se répartiront en voix.\\




Les formes rythmiques devront être distribuées dans 4 grandes
catégories
\cite{system_drums}~ :
\begin{table}[h]
\centering
\begin{tabular}{|c|c|c|c|c|} \hline
FR & SR\footnotemark & Subdivisions & Possibles & nb
voix \\ \hline
binaires & simple & doubles-croches & triolets, sextolets & 2 \\
jazz & simple & triolets & croches et doubles & 2 \\
ternaires & complexe & croches & duolets, quartelets & 2 \\
afros-cubains & simple & croches & - & 3 \\ \hline
\end{tabular}
\caption{Les formes rythmiques}
\end{table}\\
\footnotetext{SR du tableau signifie «~signature rythmique~»}
Nous exposerons 3 formes rythmiques afin d’illustrer les propos de
cette section~ :
\begin{itemize}
	\item 4/4 binaire 
	\item 4/4 jazz
	\item 4/4 afro-cubain
\end{itemize}

\subsubsection{Détection de la signature rythmique}
La partie motif des FR sera utilisée pour la
\textbf{définition des signature rythmiques}. 
\label{sign_rythm}
La détection de la signature rythmique est importante, 
non seulement pour connaître le nombre de temps par mesure ainsi que le nombre
de subdivisions pour chacun de ces temps, mais aussi pour savoir comment écrire
l’unité de temps et ses subdivisions.

\begin{figure}[h]
	\centering
	\includegraphics[height=40mm, width=40mm]{
    z_images/3_methodes/2_systemes/0_simple_VS_complexe.png}
	\caption{Les signatures rythmiques}
	\label{subdivisions}
\end{figure} %\newpage

La figure \ref{subdivisions} montre deux signatures rythmiques différentes. 
L’une (exemple 1) est \textit{simple} (2 temps binaires sur lesquels sont joués
des triolets), l’autre (exemple 2) est \textit{complexe} (2 temps ternaires). 
Le jazz est traditionnellement écrit en binaire avec ou sans triolet (même si
cette musique est dite ternaire alors que le rock ternaire sera plutôt écrit
comme dans l’exemple 2).

\subsubsection{Choix d’une grammaire}
Il faut prendre en compte l’existence potentielle de plusieurs grammaires
qui regroupent les contenus MIDI par signature rythmique. Le choix d’une
grammaire pondérée doit être fait avant le parsing puisque qparse prend en
entrée un fichier MIDI et un fichier wta (grammaire). C’est pour cette raison
que la signature rythmique doit être définie avant le choix de la grammaire.
Il faudrait les trouver automatiquement sans autres indications que les
contenus MIDI. Par conséquent, les motifs des
formes rythmiques devront être recherchés sur l’input (fichiers MIDI)
avant le lancement du parsing, afin de déterminer la signature rythmique en
amont. Cette tâche devra probablement être effectuée par utilisation
d’apprentissage automatique.


\subsubsection{Séparation des voix}

\label{sys_sep_voix}
\begin{figure}[h]
	\centering
	\includegraphics[height=60mm, width=40mm]{
    z_images/3_methodes/2_systemes/1_separation_4-4_binaire.png}
	\caption{Forme rythmique 4/4 binaire}
	\label{binaire}
\end{figure}


Ici, la forme rythmique est construite sur un modèle rock avec une
signature rythmique en 4/4.

La première ligne de la figure \ref{binaire} est
appelée «~Irréductible~» car il n’y a pas d’autre choix de séparation des voix
pour la ride et le charley au pied.

La troisième séparation proposée est privilégiée car elle répartit selon deux 
voix, une voix pour les mains (ride et caisse claire) et une voix pour les
pieds (charley et grosse caisse). Ce choix paraît plus équilibré car deux
instruments sont utilisés par voix (contrairement aux séparations possibles 1 et 2
de la figure \ref{binaire}) et plus logique pour le lecteur puisque les mains
sont en haut et les pieds en bas.

\begin{figure}[h]
\centering
\includegraphics[height=45mm, width=60mm]{
z_images/3_methodes/2_systemes/2_separation_4-4_jazz.png}
\caption{Forme rythmique 4/4 jazz}
\label{jazz}
\end{figure}
\newpage
Dans la plupart des méthodes, le charley n’est pas écrit car il est considéré
comme évident en jazz traditionnel. Ici, le parti pris est de tout écrire.

Dans la figure \ref{jazz}, les mesures 1 et 2 de la ligne «~Exemple~» combinées
avec le motif de la première ligne, sont des cas typiques de la
batterie jazz. Tout mettre sur la voix haute serait surchargé. De plus, la
grosse caisse entre très souvent dans le flot des combinaisons de toms et de
caisse claire et son écriture séparée serait inutilement compliquée et peu
intuitive pour le lecteur. Le choix de séparation sera donc de laisser les
cymbales jouées à la main en haut, et les toms, la caisse claire, la grosse
caisse et la pédale de charley en bas.

\begin{figure}[h]
	\centering
	\includegraphics[height=20mm, width=70mm]{
    z_images/3_methodes/2_systemes/3_separation_afro-latins.png}
	\caption{Forme rythmique 4/4 afro-latin}
	\label{afro_latin}
\end{figure}

La figure \ref{afro_latin} montre un exemple minimaliste de
forme rythmique afro-latin \cite{system_drums}.

Cette forme rythmique doit être écrite sur trois voix car la voix
centrale est souvent plus complexe que sur la figure et la mélanger avec le
haut ou le bas serait surchargé et peu lisible.

\subsubsection{Simplification de l’écriture}

Les règles de simplification (les combinaisons de réécritures) seront
extraites des voix séparées des formes rythmiques.
Les explications qui suivent seront appuyées par une réécriture guidée dans la
section \ref{reecriture_guidee}.

Les gammes qui accompagnent les motifs étayent toutes les
combinaisons d’un FR et elles permettent, combinées avec le
motif, de définir ses propres règles de simplification.\\

Voici les différentes étapes à suivre~ :
\begin{itemize}
	\item pour chaque gamme d’une forme rythmique, faire un
        arbre de rythmes représentant la gamme combinée avec le
        motif de la FR~ ;
	\item pour chaque arbre de rythmes obtenus, séparer les voix et faire un
        arbre de rythmes par voix~ ;
	\item pour chaque voix (arbres de rythmes) obtenus, extraire tous les nœuds
        qui nécessitent une simplification et écrire la règle.\\
\end{itemize}

Certaines précisions concernant l’extraction de ces règles sont nécessaires. 
Il s’agit de précisions à propos de la durée, des silences et de la présence ou
non d’ouvertures de charley dans les instruments joués. 
Nous avons discuté de ces problèmes plus haut dans ce chapitre.\\

Voici quelques règles inhérentes à la simplication de l’écriture pour la
batterie~ :

Même si on favorise l’usage des silences pour l’écart entre les notes
n’appartenant pas au même temps, on les remplace systèmatiquement par un point
pour 2 notes au sein d’un même temps.

\begin{figure}[h]
	\centering
	\includegraphics[height=45mm, width=50mm]{
    z_images/3_methodes/2_systemes/simplification_0.png}
	\includegraphics[height=50mm, width=40mm]{
    z_images/3_methodes/2_systemes/simplification_2.png}
	\caption{Simplifications — arbres et notations possibles}
	\label{simpl}
\end{figure}

Dans la figure \ref{simpl}, les «~suites~» sont des notations possibles
relatives aux arbres 1 ou 2.\\

\textit{Rappel~ :\\cf = charley fermé joué à la main~ ;\\co = charley ouvert joué
à la main~ ;\\ pf = charley fermé joué au pied.}\\

Soit l’arbre 1 de la figure \ref{simpl} dans lequel~ :
\begin{itemize}
    \item a et d sont des instruments de la batterie (x)~ ;
    \item b et c sont des continuations (t).
\end{itemize}
Pour chacune des conditions suivantes, une suite de la
figure \ref{simpl} est attribuée~ :
\begin{itemize}
	\item Si a n’est pas un co~ :\\
	$\Rightarrow$ Suite 1a.
	\item Si a est un co~ :
	\begin{itemize}
		\item Si d est un cf~ :\\
		$\Rightarrow$ Suite 2a.
		\item Si d est un pf~ :\\
		$\Rightarrow$ Suite 3a~ : d deviens un silence (r).\\
	\end{itemize}
\end{itemize}
Soit l’arbre 2 de la figure \ref{simpl} dans lequel~ :\\
a et c sont des instruments de la batterie (x)~ ;\\
b est une continuation (t)~ ;
Pour chacune des conditions suivantes, une suite de la figure \ref{simpl} est
attribuée~ :
\begin{itemize}
	\item Si a n’est pas un co~ :\\
	$\Rightarrow$ Suite 1b, b devient un silence.
	\item Si a est un co~ :
	\begin{itemize}
		\item Si c est un cf~ :\\
		$\Rightarrow$ Suite 2b, b devient une liaison et c devient un cf.
		\item Si c est un pf~ :\\
		$\Rightarrow$ Suite 3b~ : b deviens une liaison et c devient un silence.
	\end{itemize}
\end{itemize}

\section*{Conclusion}
Dans ce chapitre, nous avons formalisé une notation de la batterie inspirée des
méthodes de batterie Dante Agostini, modélisé cette notation pour
la transcription de données MIDI en partition.
Nous avons ensuite parlé de l’outil utilisé pour les transcriptions manuelles
en mettant en avant que cet outil devrait être utilisé pour la transcription de
la batterie et en précisant que tous les codes lilypond pour la création des
figures et partition de ce mémoire sont en accès libre sur github. Nous avons
ensuite décrit qparse qui est l’outil que le travail de ce mémoire cherche à
améliorer.

Enfin, nous avons exposé une approche de type dictionnaire (les «~formes
rythmiques~») pour détecter une signature rythmique, choisir une grammaire
pondérée appropriée et énoncer des règles de séparation des voix et de
simplification de l’écriture.
