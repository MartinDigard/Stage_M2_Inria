\section*{Introduction}
Dans ce chapitre, nous expliquerons en détail les méthodes que nous avons employées pour l’ADT.\\
Pour commencer, nous exposerons une description de la notation de la batterie ainsi qu’une modélisation de celle-ci pour la représentation des données rythmiques en arbres syntaxiques. Nous poursuiverons avec une présentation de qparse\footnote{https://qparse.gitlabpages.inria.fr/}, un outil de transcription qui est développé par Florent Jacquemard (Inria) au sein du laboratoire Cedric au CNAM.\\
Enfin, nous présenterons les systèmes. 
\section{La notation de la batterie}
\label{notation_batterie}
\begin{figure}[h]
	\centering
	\includegraphics[height=10mm, width=25mm]{z_images/3_methodes/0_notation_de_la_batterie/0_figures_de_notes.png}
\end{figure}
Une figure de note \cite{danhauser} de musique combine plusieurs critères\footnote{\url{https://fr.wikipedia.org/wiki/Note_de_musique}} :
\begin{itemize}
	\item Une tête de note :\\
	Sa position sur la portée indique la hauteur de la note. La tête de note peut aussi indiquer une durée.
	\item Une hampe :\\
	Indicatrice d’appartenance à une voix en fonction de sa direction et indicatrice d’une durée représentée par sa présence ou non (blanche ≠ ronde)
	\item Un crochet : La durée d’une note est divisée par deux à chaque crochet ajouté à la hampe d’une figure de note.
\end{itemize}
\begin{figure}[h]
	\centering
	\includegraphics[height=50mm, width=80mm]{z_images/3_methodes/0_notation_de_la_batterie/1_rapport_figures_notes.png}
	\caption{Rapport des figures de notes}\cite{danhauser}
	\label{rapp_fig_notes}
\end{figure}
La figure \ref{rapp_fig_notes} montre les rapports de durée entre les figures de notes. Plus les durées sont longues, plus elles sont marquées par la tête de note (la note carrée fait deux fois la durée d’une ronde) ou la présence ou non de la hampe. À partir de la noire (3ème lignes en partant du haut), on ajoute un crochet à la hampe d’une figure de notes pour diviser sa durée par 2. Les notes à crochet (croche , double-croche, triple…) peuvent être reliées ou non par des ligatures (Voir les 4 dernière lignes de la figure \ref{rapp_fig_notes}).
\subsection*{Les hauteurs et les têtes de notes}
Pour la transcription, nous proposons une notation inspirée du recueil de pièces pour batterie de J.-F. Juskowiak \cite{jusko} et des méthodes de batterie Agostini \cite{ago_meth_3}, car nous trouvons la position des éléments cohérente et intuitive.\\
En effet, les hauteurs sur la portée représentent :
\begin{itemize}
	\item La hauteur physique des instruments :\\
	La caisse claire est centrale sur la portée et sur la batterie (au niveau de la ceinture, elle conditionne l’écart entre les pédales et aussi la position de tous les instruments basiques d’une batterie).\\
	Tout ce qui en-dessous de la caisse-claire sur la portée est en dessous de la caisse-claire sur la batterie (pédales, tom basse) ;\\
	Tout ce qui est au-dessus de la caisse-claire sur la portée, l’est aussi sur la batterie.\\
	\item La hauteur des instruments en terme de fréquences :\\
	Sauf pour le charley au pied et si l’on sépare en trois groupes (grosse-caisse, toms et cymbales), de bas en haut, les instruments vont du plus grave au plus aigu.
\end{itemize}
\begin{figure}[!h]
	\centering
	\includegraphics[height=25mm, width=130mm]{z_images/3_methodes/0_notation_de_la_batterie/2_hauteurs_et_tete_de_notes.png}
	\caption{Hauteur et têtes de notes}
	\label{Hauteur et têtes de notes}
\end{figure}
Les noms des instruments correspondant aux codes des notes de la figure \ref{Hauteur et têtes de notes} sont dans le tableau \ref{pitchs_instru}.
\subsection*{Les durées}
\label{hho}
Comme nous venons de la voir, la majorité des instruments de la batterie sont représentés par les têtes des notes. Par conséquent, les symboles rythmiques concernant la tête de note ne pourront pas être utilisés. Cela est valable aussi pour la présence ou non de la hampe puisque ce phénomène n’existe qu’avec les têtes de notes de type cercle-vide (opposition blanche-ronde). L’usage des blanches existe dans certaines partitions de batterie \cite{system_drums} mais cela reste dans des cas très rares. Certains logiciels permettent de faire des blanches avec des symboles spécifiques à la batterie ou aux percussions mais leur lecture reste peu aisée et leur utilisation pour la batterie est rarissime.\\
La durée d’une note peut être allongée par divers symboles :
\begin{itemize}
	\item Le point ;
	\item La liaison.
\end{itemize}
Ces symboles ne seront utiles que pour l’écriture des ouvertures de charley. Le charley est le seul instrument de la batterie dont la durée est quantitifiée (les cymbales attrapées à la main peuvent l’être aussi mais cela est très rare.)\newpage
\begin{figure}[h]
	\centering
	\includegraphics[height=80mm, width=40mm]{z_images/3_methodes/0_notation_de_la_batterie/3_point_et_liaison.png}
	\caption{Point et liaison}
	\label{point_liaison}
\end{figure}
L’écriture de la batterie doit faire ressortir la pulsation. La première chose à prendre en compte pour analyser la figure \ref{point_liaison} est donc la nécessité de regrouper les notes par temps à l’aide des ligatures.\\
Exemple 1 : ouverture de charley quantifiée mais pas notes pas regroupées par temps.\\
Exemple 2 : bieeen !\\
Exemple 3 et exemple 4 : les deux exemples sont valables mais le deuxième est le plus souvent utilisé car plus intuitif (regroupement par temps).\\
En cas de nécessité de rallonger la durée d’une note pour la batterie, on privilégiera la liaison.
\subsection*{Les silences}
\begin{figure}[h]
	\centering
	\includegraphics[height=35mm, width=120mm]{z_images/3_methodes/0_notation_de_la_batterie/4_silences.png}
	\caption{Les silences}
	\label{silences}
\end{figure}\newpage
Les silences sont parfois utilisés pour quantifier les ouvertures de charley. Les fermetures du charley sont notées soit par un silence (correspondant à une fermeture de la pédale), soit par un écrasement de l’ouverture par un autre coup de charley fermé, au pied ou à la main.
Physiquement, le charley est fermé par une pression du pied sur la pédale de charley. Dans les fichiers MIDI, cette pression est traduite par un charley joué au pied. Mais dans une vraie partition, cette écriture ne traduirait pas ce que le batteur doit penser.
\begin{figure}[h]
	\centering
	\includegraphics[height=40mm, width=40mm]{z_images/3_methodes/0_notation_de_la_batterie/5_silence_joue.png}
	\caption{Silence joué}
	\label{silence joue}
\end{figure}\\
L’exemple 1 de la figure \ref{silence joue} montre ce qui est écrit dans les données MIDI et l’exemple 2 montre ce que le batteur doit penser en lisant la partition. Il faut aussi prendre en compte l’écriture surchargée que l’exemple 1 donnerait avec une partition comprenant plusieurs voix et plusieurs instruments jouant simultanément.
\subsection*{Les équivalences rythmiques}
Pour les instruments mélodiques, la liaison et le point sont les deux seules possibilités en cas d’équivalence rythmique pour des notes dont la durée de l’une à l’autre est ininterrompue. Mais pour la batterie, à part pour les ouvertures de charley (voir section \ref{hho}), les durées des notes n’ont pas d’importance. L’usage des silences pour combler la distance rythmique entre deux notes devient donc possible.\\
Cela pris en compte, et étant donné que les indications de durée dans les têtes de notes sont peu recommandées (voir section \ref{hho}), l’écriture à l’aide de silences sera privilégiée comme indication de durée sauf dans les cas où cela reste impossible. Ce choix à pour but de n’avoir qu’une manière d’écrire toutes les notes, que leurs têtes de notes soit modifiées ou non.
\begin{figure}[h]
	\centering
	\includegraphics[height=20mm, width=75mm]{z_images/3_methodes/0_notation_de_la_batterie/6_equivalence.png}
	\caption{Équivalence}
	\label{equivalence}
\end{figure}\\
Sur la figure \ref{equivalence}, théoriquement, il faudra choisir la notation de la deuxième mesure mais dans certains contextes, pour des raisons de lisibilité ou de surcharge, la version sans les silences de la troisième mesure pourra être choisie.
\subsection*{Les voix}
Les voix\footnote{\url{https://fr.wikipedia.org/wiki/Voix_(polyphonie)}} désignent les différentes parties mélodiques constituant une composition musicale et destinées à être interprétées, simultanément ou successivement, par un ou plusieurs musiciens. En batterie, une voix est l’ensemble des instruments qui, à eux seuls, constituent une phrase rythmique et sont regroupés à l’aide des ligatures. Plusieurs écritures étant possibles pour un même rythme, on peut regrouper les instruments de la batterie par voix. Sur une portée de batterie, il existe le plus souvent 1 ou 2 voix. Sur la figure \ref{sep_voix}, il faudra faire un choix entre les exemples 1, 2 et 3 qui sont trois façons d’écrire le même rythme.
\begin{figure}[h]
	\centering
	\includegraphics[height=65mm, width=60mm]{z_images/3_methodes/0_notation_de_la_batterie/7_voix.png}
	\caption{Séparation des voix}
	\label{sep_voix}
\end{figure}\\
Ce choix se fera en fonction des instruments joués, de la nature plus ou moins systèmatique de leurs phrasés, et des associations logiques entre les instruments dans la distribution des rythmes sur la batterie (voir la section \ref{sys_sep_voix}).
\subsection*{Les accentuations et les ghost-notes}
« Certaines notes dans une phrase musicale doivent, ainsi que les différentes syllabes d’un mot, être accentuées avec plus ou moins de force, porter une inflexion particulière. » \cite{danhauser}
\begin{figure}[h]
	\centering
	\includegraphics[height=25mm, width=75mm]{z_images/3_methodes/0_notation_de_la_batterie/8_nuances_0.png}
	\caption{Les accents et les ghost-notes}
	\label{accents_et_gn}
\end{figure}\\
La figure \ref{accents_et_gn} ne prend en compte que les accents que nous avons estimés nécessaires (voir la section \ref{velocite}). Les accents sont marqués par le symbole « > ». Il est positionné au-dessus des notes représentant des cymbales et en-dessous des notes représentant des toms ou la caisse-claire. Ce choix a été fait pour la partition de la figure \ref{partition_ref} car elle est plus lisible ainsi, mais ces choix devront être adaptés en fonction des différents systèmes reconnus (voir la section \ref{systemes_methodes}). Par exemple, pour les systèmes jazz, les ligatures pour les toms et la caisse-claire seront dirigés vers le bas, il faudra donc mettre les symboles d’accentuation correspondants au-dessus des têtes de notes.\\
La dernière note de la figure \ref{accents_et_gn} montre un exemple de ghost-notes. Le parenthésage a été choisi car il peut être utilisé sur n’importe quelle note sans changer la tête de note.\\
Pour les codes, on prend le code de la note et on ajoute un « a » pour un accent et un « g » pour une ghost-note. Toutes les notes de la figure \ref{accents_et_gn} sont exposées en situation réelle dans la figure \ref{exemple_acc_et_gn}. 
\begin{figure}[h]
	\centering
	\includegraphics[height=20mm, width=110mm]{z_images/3_methodes/0_notation_de_la_batterie/8_nuances_1.png}
	\caption{Exemple pour les accentuations et les ghost-notes}
	\label{exemple_acc_et_gn}
\end{figure}\newpage
\section{Modélisation pour la transcription}
\label{modelisation_transcription}
\subsection*{Les pitchs}
\begin{table}[h]
	\centering
	\begin{tabular}{|c|c|c|} \hline
		Codes & Instruments & Pitchs \\ \hline
		cf & charley-main-fermé & 22, 42 \\
		co & charley-main-ouvert & 26 \\
		pf & charley-pied-fermé & 44 \\
		rd & ride & 51 \\
		rb & ride-cloche (bell) & 53 \\
		rc & ride-crash & 59 \\
		cr & crash & 55 \\
		cc & caisse-claire & 38, 40 \\
		cs & cross-stick & 37 \\
		ta & tom-alto & 48, 50 \\
		tm & tom-medium & 45, 47 \\
		tb & tom-basse & 43, 58 \\
		gc & grosse-caisse & 36 \\ \hline
	\end{tabular}
	\caption{Pitchs et instruments}
	\label{pitchs_instru}
\end{table}
Il existe, pour de nombreux instruments de la batterie, plusieurs samples audio associés à des pitchs. Pour cette première version, nous avons choisi de n’avoir qu’un code-instrument pour différentes variantes d’un instrument, c’est pourquoi certain code-instrument se voit attribuer plusieurs pitchs dans le tableau \ref{pitchs_instru}.\\
Malgré le large panel de pitchs disponible, il semblerait qu’aucun pitch ne désigne le charley ouvert joué au pied. Pourtant, dans la batterie moderne, plusieurs rythmes ne peuvent fournir le son du charley ouvert qu’avec le pied car les mains ne sont pas disponibles pour le jouer. Cela doit en partie être dû à l’utilisation des boîte à rythmes en MAO qui ne nécessitent pas de faire des choix conditionnés par les limitations humaines (2 pieds, 2 mains, et beaucoup plus d’instruments…)
\subsection*{La vélocité}
\label{velocite}
La partition de la figure \ref{partition_ref} a été transcrite manuellement avec lilypond par analyse des fichiers MIDI et audio correspondants.\\
Cette transcription nous a mené aux observations suivantes :
\begin{itemize}
	\item Vélocité inférieure à 40 : ghost-note ;
	\item Vélocité supérieure à 90 : accent ;
	\item Pas d’intention d’accent ni de ghost-note pour une vélocité entre 40 et 89 ;
	\item Les accents et les ghosts-notes ne sont significatifs ni pour les instruments joués au pied, ni pour les cymbales crash.\\
	En effet, certaines vélocités en dessous de 40 étant détectées et inscrites dans les données MIDI sont dues au mouvement du talon du batteur qui bat la pulsation sans particulièrement jouer le charley. Ce mouvement est perçu par le capteur de la batterie électronique mais le charley n’est pas joué.
	\item Au final, nous avons relevé les ghost-notes et les accents pour la caisse-claire ainsi que les accents pour les toms et les cymbales rythmiques (charley et ride).
\end{itemize}
\subsection*{Les arbres de rythmes}
Les arbres de rythmes représentent un rythme unique dont les possibilités de notation sur une partition sont théoriquement multiples. Voici une représentation de la figure \ref{sep_voix} en arbre de rythmes avec les codes de chaque instrument :
\begin{figure}[h]
	\Tree[ [ [rd\\gc ][ [rd\\pf ][rd ]]]
	[ [rd\\cc ][ [rd\\pf ][rd ]]]
	[ [rd\\gc ][ [rd\\pf ][rd ]]]
	[ [rd\\cc ][ [rd\\pf ][rd ]]] ]
\end{figure}\\
Ci-dessous, le même arbre dont les codes des instruments sont remplacés par leurs données MIDI respectives :
\begin{figure}[h]
	\Tree[ [ [51\\36 ][ [51\\44 ][51 ]]]
	[ [51\\38 ][ [51\\44 ][51 ]]]
	[ [51\\36 ][ [51\\44 ][51 ]]]
	[ [51\\38 ][ [51\\44 ][51 ]]] ]
\end{figure}\\
Chacun des trois exemples de la figure \ref{sep_voix} est représenté par un des deux arbres syntaxiques ci-dessus.\newpage
\section{Qparse}
La librairie Qparse\footnote{\url{https://qparse.gitlabpages.inria.fr}} implémente la quantification des rythmes basée sur des algorithmes d'analyse syntaxique pour les automates arborescents pondérés. En prenant en entrée une performance musicale symbolique (séquence de notes avec dates et durées en temps réel, typiquement un fichier MIDI), et une grammaire hors-contexte pondérée décrivant un langage de rythmes préférés, il produit une partition musicale. Plusieurs formats de sortie sont possibles, dont XML MEI.
Les principaux contributeurs sont :
\begin{itemize}
	\item Florent Jacquemard (Inria) : développeur principal.
	\item Francesco Foscarin (PhD, CNAM) : construction de grammaire automatique à partir de corpus ; Evaluation.
	\item Clement Poncelet (Salzburg U.) : integration de la librairie Midifile pour les input MIDI.
	\item Philippe Rigaux (CNAM) : production de partition au format MEI et de modèle intermédiaire de partition en sortie.
	\item Masahiko Sakai (Nagoya U.) : mesure de la distance input/output pour la quantification et CMake framework ; évaluation.
\end{itemize}
\begin{figure}[h]
	\centering
	\includegraphics[height=95mm, width=130mm]{z_images/3_methodes/1_qparse/0_general_qparse.png}
	\caption{Présentation de Qparse}
	\label{presentation_qparse}
\end{figure}
Explication des différentes étapes de la figure \ref{presentation_qparse}\footnote{\url{https://gitlab.inria.fr/qparse/qparselib/-/tree/distance/src/scoremodel}} :
\begin{itemize}
	\item \textbf{Input Qparse} :\\
	Un fichier MIDI (séquence d’événements datés (piano roll) accompagné d’un fichier contenant une grammaire pondérée) ;
	\item \textbf{Arbre de parsing} :\\
	Les données MIDI sont quantifiées, les notes de dates proches sont alignées et les relations entre les notes sont identifiées (accords, fla, etc…) ; un arbre de parsing global est créé ;
	\item \textbf{Score Model} :
	\begin{itemize}
		\item Les instruments sont identifiés dans scoremodel/import/tableImporterDrum.cpp ;
		\item Réécriture 1 :\\
		séparation des voix $\Rightarrow$ un arbre par voix $\Rightarrow$ représentation intermédiaire (RI) ;
		\item Réécriture 2 :\\
		simplification de l’écriture de chaque voix dans la RI ;
	\end{itemize}
    \item \textbf{Output} :\\
    export de la partition. Plusieurs formats sont possibles (xml, mei, lilypond,… ).\\
\end{itemize}
Plusieurs enjeux :
\begin{itemize}
	\item Problème du MIDI avec Qparse :\\
	ON-OFF en entrée $\Rightarrow$ 1 seul symbole en sortie.
	\item Minimiser la distance entre le midi et la représentation en arbre.
	\item Un des problèmes de Qparse était qu’il était limité au monophonique.\\
	Quelles sont les limites du monophonique ?
	\begin{itemize}
		\item Impossibilité de traiter plusieurs voix et de reconnaître les accords.
	\end{itemize}
\end{itemize}
\section{Les systèmes}
\label{systemes_methodes}
Un système est la combinaison d’un ou de plusieurs éléments qui jouent un rythme en boucle (motif) et d’un autre élément qui joue un texte rythmique variable mais en respectant les règles propres au système (gamme).
\subsection*{Définitions}
\textit{\textbf{Système :}} motif + gamme/texte\\
\textit{\textbf{Motif :}} rythmes coordonnés joués avec 2 ou 3 membres en boucle (répartis sur 1 ou 2 voix)\\
\textit{\textbf{Texte :}} rythme irrégulier joué avec un seul membre sur le motif (réparti sur 1 voix).\\
\textit{\textbf{Gamme :}} la gamme d’un système considère l’ensemble des combinaisons que le batteur pourrait rencontrer en interprétant un texte rythmique à l’aide du système.\\\\
Un ensemble de systèmes comprenant leur métrique et leurs règles spécifiques de réécriture sera nécessaire. Les systèmes devront être distribués dans 4 grandes catégories :
\begin{table}[h]
	\centering
	\begin{tabular}{|c|c|c|c|c|} \hline
		Systèmes & Métriques & Subdivisions & Possibles & nb voix \\ \hline
		binaires & simple & doubles-croches & triolets, sextolets & 2 \\
		jazz & simple & triolets & croches et doubles-croches & 2 \\
		ternaires & complexe & croches & duolets, quartelets & 2 \\
		afros-cubains & simple & croches & - & 3 \\ \hline
	\end{tabular}
	\caption{Sytèmes}
\end{table}\\
Nous exposerons 3 systèmes afin d’illustrer les propos de cette section :
\begin{itemize}
	\item 4/4 binaire 
	\item 4/4 jazz
	\item 4/4 afro-cubain
\end{itemize}
\subsection*{Objectif des systèmes}
Les systèmes devront être matchés sur l’input MIDI afin de :
\begin{itemize}
	\item définir une métrique ;
	\item choisir une grammaire appropriée ;
	\item fournir les règles de réécriture (séparation des voix et simplification.\\
\end{itemize}
La partie \textit{motif} des systèmes sera utilisée pour la \textbf{définition des métriques}. Le \textit{motif} et la gammes des systèmes seront utilisés pour la \textbf{séparation des voix}. Les règles de \textbf{simplification} (les combinaisons de réécritures) seront extraites des voix séparées des systèmes.
\subsubsection{Détection d’indication de mesure}
La détection de la métrique est importante, non seulement pour connaître le nombre de temps par mesure ainsi que le nombre de subdivisions pour chacun de ces temps, mais aussi pour savoir comment écrire l’unité de temps et ses subdivisions.
\begin{figure}[h]
	\centering
	\includegraphics[height=40mm, width=40mm]{z_images/3_methodes/2_systemes/0_simple_VS_complexe.png}
	\caption{Métrique}
	\label{subdivisions}
\end{figure}\newpage
La figure \ref{subdivisions} montre deux indications de mesure différentes. L’une (exemple 1) est \textit{simple} (2 temps binaires sur lesquels sont joués des triolets), l’autre (exemple 2) est \textit{complexe} (2 temps ternaires). Le jazz est traditionnellement écrit en binaire avec ou sans triolet (même si cette musique est dite ternaire alors que le rock ternaire sera plutôt écrit comme dans l’exemple 2).
\subsubsection{Choix d’une grammaire}
Il faut prendre en compte l’existence potentielle de plusieurs grammaires dédiées chacunes à un type de contenu MIDI. Le choix d’une grammaire pondérée doit être fait avant le parsing puisque Qparse prend en entrée un fichier MIDI et un fichier wta (grammaire). C’est pour cette raison que la métrique doit être définie avant le choix de la grammaire.\\
Pour les expériences effectuées avec le Groove MIDI Data Set, le style et l’indication de mesure sont récupérables par les noms des fichiers MIDI, mais il faudra par la suite les trouver automatiquement sans autres indications que les données MIDI elles-mêmes. Par conséquent, les motifs des systèmes devront être recherchés sur l’input \textit{(fichiers MIDI)} avant le lancement du parsing, afin de déterminer la métrique en amont. Cette tâche devra probablement être effectuée en Machine Learning.\newpage
\subsubsection{Séparation des voix}
\label{sys_sep_voix}
\begin{figure}[h]
	\centering
	\includegraphics[height=60mm, width=40mm]{z_images/3_methodes/2_systemes/1_separation_4-4_binaire.png}
	\caption{Motif 4-4 binaire}
	\label{binaire}
\end{figure}
Ici, le système est construit sur un modèle rock en 4/4 : after-beat sur les 2 et 4 avec un choix de répartition des cymbales type fast-jazz. Le système est constitué par défaut du motif rd/pf/cc (voir \ref{pitchs_instru}) et d’un texte joué à la grosse-caisse. La première ligne de la figure \ref{binaire} est appelée « Irréductible » car il n’y a pas d’autre choix pertinent pour la répartition de la ride et du charley au pied. La troisième séparation proposée est privilégiée car elle répartit selon 2 voix, une voix pour les mains (rd + cc) et une voix pour les pieds (pf + gc). Ce choix paraît plus équilibré car deux instruments sont utilisés par voix et plus logique pour le lecteur puisque les mains sont en haut et les pieds en bas.
\begin{figure}[h]
	\centering
	\includegraphics[height=45mm, width=60mm]{z_images/3_methodes/2_systemes/2_separation_4-4_jazz.png}
	\caption{Motif 4-4 jazz}
	\label{jazz}
\end{figure}\\
Dans la plupart des méthodes, le charley n’est pas écrit car il est considéré comme évident en jazz traditionnel. Ce qui facilite grandement l’écriture : la ride et les crash sur la voix du haut et le reste sur la voix du bas. Ici, le parti pris est de tout écrire. Dans l’exemple ci-dessus, les mesures 1 et 2 combinées avec le \textit{motif} de la première ligne, sont des cas typiques de la batterie jazz. Tout mettre sur la voix haute serait surchargé. De plus, la grosse caisse entre très souvent dans le flot des combinaisons de toms et de caisse claire et son écriture séparée serait inutilement compliquée et peu intuitive pour le lecteur. Le choix de séparation sera donc de laisser les cymbales en haut et toms, caisse-claire, grosse-caisse et pédale de charley en bas.
\begin{figure}[h]
	\centering
	\includegraphics[height=20mm, width=70mm]{z_images/3_methodes/2_systemes/3_separation_afro-latins.png}
	\caption{Système 4-4 afro-latin}
	\label{afro_latin}
\end{figure}\\
La figure \ref{afro_latin} montre un exemple minimaliste de système afro-latin \cite{system_drums}. Ce système doit être écrit sur trois voix car la voix centrale est souvent plus complexe qu’ici (que des noirs) et la mélanger avec le haut ou le bas serait surchargé et peu lisible.
\subsubsection{Simplification de l’écriture}
Les gammes qui accompagnent les motifs d’un système étayent toutes les combinaisons d’un système. Elles sont générées manuellement à partir de la simplification de chaque voix d’un système selon les principes de la section \ref{notation_batterie} (aller à la section \ref{demo_sys} pour voir un exemple). 
\section*{Conclusion}
Nous avons formalisé une notation de la batterie, modélisé cette notation pour la transcription de données MIDI en partition, nous avons décrit Qparse.\\
Enfin, nous avons exposé une approche de type dictionnaire (les « systèmes ») pour détecter une métrique, choisir une grammaire pondérée appropriée et énoncer des règles de séparation des voix et de simplification de l’écriture.