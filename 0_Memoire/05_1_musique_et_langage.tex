\section{La musique et le TAL}
Aborder la musique à travers le TAL nécessite une réflexion autour de la musique en tant que langage ainsi que la possibilité de comparer ce même langage avec les langues naturelles. Quelques travaux en neuroscience ont abordé la question, notamment par observation des processus cognitifs et neuronaux que les systèmes de traitement de ces deux langages avaient en communs. Dans le travail de Poulin-Charronnat et al. \cite{poulincharronnat:hal-01985213}, la musique est reconnue comme étant un système complexe spécifique à l’être humain dont une des similitudes avec les langues naturelles est l’émergence de régularités reconnues implicitement par le système cognitif. La question de la pertinence de l’analogie entre langues naturelles et langage musical a également été soulevée à l’occasion de projets de recherche en TAL. Keller et al. \cite{keller:hal-03279850} ont exploré le potentiel de ces techniques à travers les plongements de mots et le mécanisme d’attention pour la modélisation de données musicales. La question du sens d’une phrase musicale apparaît, selon eux, à la fois comme une limite et un défi majeur pour l’étude de cette analogie.\\
D’autre travaux très récents, ont aussi été révélé lors de la \textit{première conférence sur le NLP pour la musique et l'audio (NLP4MusA 2020)}. Lors de cette conférence, Jiang et al. \cite{Jiang2020DiscoveringMR} ont présenté leur implémentation d’un modèle de langage musical auto-attentif visant à améliorer le mécanisme d'attention par élément, déjà très largement utilisé dans les modèles de séquence modernes pour le texte et la musique.
\section{La transcription automatique de la musique}
L'objectif de la transcription automatique de la musique (AMT) \cite{article1} est de convertir la performance d'un musicien en notation musicale - un peu comme la conversion de la parole en texte dans le traitement du langage naturel. Bien que l’AMT soit un domaine de recherche en plein essor dans lequel plusieurs approches différentes sont encore activement étudiées, les performances des systèmes actuels ne sont pas encore suffisantes pour certaines applications qui exigent un haut degré de précision \cite{article1}. Même si les applications typiques de l'AMT comprennent l'estimation de la multi-tonalité, la classification des genres musicaux, la détection du début et de la fin des notes de musique, l'estimation du tempo, le suivi du rythme et la transcription de la musique. La plupart des travaux se sont concentrés sur le traitement du signal vers la génération du midi \cite{article2}. Seuls quelques travaux récents \cite{foscarin:hal-01988990} s’intéressent de près à la création d’outils permettant la génération de partition. Les applications de l’AMT ont aussi de la valeur dans les domaines oraux ou d’improvisation qui manquent de partition (jazz, pop) \cite{article1}. La batterie étant amplement représentée dans ces styles de musique, l’ADT serait utile.
\section{La transcription automatique de la batterie}
La batterie est un instrument récent qui s’est longtemps passé de partition. En effet, la qualité de lecteur en tant que batteur, lorsqu’elle était nécessaire, résidait essentiellement dans sa capacité à lire les partitions des autres instrumentistes afin d’improviser un accompagnement approprié (par exemple, en jazz, les grilles d’accord et la mélodie du thème pour jouer les syncopes clés). Les partitions de batterie sont arrivées par nécessité avec la pédagogie et l’émergence d’école de batterie partout dans le monde. La musique assistée par ordinateur (MAO), à aussi largement contribué à l’expansion des partitions de batterie car l’autonomie musiciens pour écouter le résultats de leur compositions sans nécessairement avoir les musiciens pour jouer leur a permis d’écrire ce qu’il voulait entendre pour leur futures batteur. 
Si les ordinateurs étaient capables d'analyser la partie de la batterie dans la musique enregistrée, cela permettrait une variété de tâches de traitement de la musique liées au rythme. En particulier, la détection et la classification des événements sonores de la batterie par des méthodes informatiques est considérée comme un problème de recherche important et stimulant dans le domaine plus large de la recherche d'informations musicales \cite{8350302}. Pour mieux comprendre la pratique des systèmes d’ADT, Wu et al. \cite{8350302} se concentrent sur les méthodes basées sur la factorisation matricielle non négative et celles utilisant des réseaux neuronaux récurrents. Marxer et al. \cite{2802} modélisent les autres éléments du signal afin de distinguer non-pas seulement les instruments de la batterie entre eux mais aussi d’être capable de dissocier la batterie des autres instrument dans un orchestre ou un groupe.
\textit{Le cas de la transcription de la batterie (DT) est très particulier puisqu'il s'agit d'instruments sans hauteur, d'événements avec (presque) aucune durée et de notations spécifiques. Il a été la source de nombreuses études MIR, voir \cite{8350302} pour un aperçu. La plupart de ces travaux se concentrent sur des méthodes de calcul pour la détection d'événements sonores de batterie à partir de signaux acoustiques, et sur l'extraction de caractéristiques de bas niveau telles que la classe d'instrument et le moment de l'apparition du son (peak picking). Cependant, très peu d'entre eux ont abordé la tâche de générer une notation musicale (rythmique) lisible à partir des caractéristiques ci-dessus, une étape cruciale dans un contexte musical et loin d'être triviale.}
