%\subsection{Langue naturelle et langage musical}
\section{La musique et le TAL}
Aborder un sujet concernant la musique à travers le TAL nécessite une réflexion autour de la musique en tant que langage et la comparaison possible de ce même langage avec une langue naturelle. Quelques travaux en neuro-science ont abordé la question notamment par observation de processus cognitifs et neuronaux communs partagés par les systèmes de traitement par le cerveau humain de ces deux langages \cite{poulincharronnat:hal-01985213}. Dans ce travail, la musique est reconnue comme étant un système complexe spécifique à l’être humain dont une des similitudes avec les langues naturelles serait l’émergence de régularités reconnues implicitement par le système cognitif.\\
La question de la pertinence de l’analogie entre langage naturel et langage musical a également été soulevé à l’occasion de projets de recherche en TAL. Keller et al. \cite{keller:hal-03279850} ont exploré le potentiel des techniques de TALN, à travers les plongements de mots et le mécanisme d’attention, pour la modélisation de données musicales. La question du sens d’une phrase musicale apparaît, selon eux, à la fois comme une limite et un défi majeur pour l’étude de cette analogie.\\
D’autre travaux très récents, ont aussi été révélé lors de la \textit{première conférence sur le NLP pour la musique et l'audio (NLP4MusA 2020)}. Lors de cette conférence, Jiang et al. \cite{Jiang2020DiscoveringMR} ont présenté leur implèmentation d’un modèle de langage musical auto-attentif visant à améliorer le mécanisme d'attention par élément, déjà très largement utilisé dans les modèles de séquence modernes pour le texte et la musique.
\section{La transcription automatique de la musique}
\cite{article1}
\cite{article2}
voir \cite{foscarin:hal-01988990} pour des modèles structurés en arbre basés sur la théorie formelle du langage, que nous développons dans le contexte d'outils AMT plus généraux.
\section{La transcription automatique de la batterie}

