Dans ce mémoire, nous avons traité de la problématique de la transcription automatique de la batterie. Son objectif était de transcrire, à partir de leur représentation symbolique MIDI, des performances de batteur de différents niveaux et dans différents styles en partitions écrites.\\
Nous avons avancé sur le parsing des données MIDI établissant un processus de regroupement des évènements MIDI qui nous a permis de faire la transition du monophonique vers le polyphonique. Une des données importante de ce processus était de différencier les nature des notes d’un \textit{accord}, notamment de distinguer lorsque 2 notes constituent un \textit{accord} ou un \textit{fla}.\\
Nous avons établis des \textit{grammaires pondérées} pour le parsing qui correspondent respectivement à des métriques spécifiques. Celles-ci étant sélectionnables en amont du parsing, soit par indication des noms des fichiers MIDI, soit par reconnaissance de la métrique avec une approche dictionnaire de patterns prédéfinis \footnote{\textit{Motifs} dans les \textit{systèmes} de la présente proposition.} qu’il serait pertinent de mettre en œuvre en machine learning.\\
Nous avons démontré que l’usage des \textit{systèmes} élimine un grand nombre de calcul lors de la réécriture. Pour la séparation des voix grâce au motif d’un système et pour la simplification grâce aux gammes du motif d’un système. Nous avons aussi montré comment, dans des travaux futurs, un système dont le motif serait reconnu en amont dans un fichier MIDI pourrait prédéfinir le choix d’une grammaire par la reconnaissance d’une métrique et ainsi améliorer le parsing et accélérer les choix ultérieurs dans la chaîne de traitement en terme de réécriture.\\
Il sera également intéressant d'étudier comment l'utilisation de LM peut améliorer les résultats de l'AM, voir [2], et ouvrir la voie à la génération entièrement automatisée de partitions de batterie et au problème général de l'AMT de bout en bout.\cite{8350302}