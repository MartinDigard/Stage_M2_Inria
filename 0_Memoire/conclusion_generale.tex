Dans ce mémoire, nous avons traité de la problématique de la transcription
automatique de la batterie. Son objectif était de transcrire, à partir de leur
représentation symbolique MIDI, des performances de batteur de différents
niveaux et dans différents styles en partitions écrites.\\

Nous avons avancé sur le \textit{parsing} des données MIDI établissant un processus de
regroupement des évènements MIDI qui nous a permis de faire la transition du
monophonique vers le polyphonique. Une des données importante de ce processus
était de différencier les nature des notes d’un \textit{accord}, notamment de
distinguer lorsque 2 notes constituent un \textit{accord} ou un \textit{fla}.\\
Nous avons établis des \textit{grammaires pondérées} pour le \textit{parsing} qui
correspondent respectivement à des signatures rythmiques spécifiques. Celles-ci
étant sélectionnables en amont du \textit{parsing}, soit par indication des noms des
fichiers MIDI, soit par reconnaissance de la signature rythmique avec une approche dictionnaire de
patterns prédéfinis \footnote{\textit{Motifs} dans les \textit{forme rythmiques} de la
présente proposition.} qu’il serait pertinent de mettre en œuvre avec des
méthodes d’apprentissage en amont sur les données MIDI avant le \textit{parsing}.\\
Nous avons démontré que l’usage des \textit{forme rythmiques} élimine un grand nombre
de calcul lors de la réécriture. Pour la séparation des voix grâce au motif
d’une forme rythmique et pour la simplification grâce aux gammes du motif d’une forme rythmique.
Nous avons aussi montré comment, dans des travaux futurs, une forme rythmique dont le
motif serait reconnu en amont dans un fichier MIDI pourrait prédéfinir le choix
d’une grammaire par la reconnaissance d’une signature rythmique et ainsi améliorer le
\textit{parsing} et accélérer les choix ultérieurs dans la chaîne de traitement en terme
de réécriture.\\

La transcription automatique de la batterie est un sujet passionnant mais
difficile~ : Obtenir la totalité des éléments nécessaires pour le mémoire
nécessiterait plus de temps. Une base solide spécifique à la batterie a
néanmoins été générée. Elle sera un bon point de départ pour les travaux futurs
dont plusieurs propositions sont énoncés dans le présent document.
