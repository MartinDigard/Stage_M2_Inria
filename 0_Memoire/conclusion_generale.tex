\textit{\\Conclusion~: la conclusion globale du mémoire.}\\\\
Dans ce mémoire, nous avons traité de la problématique...\\\\

L'intégration de ces LM avec l'état de l'art des modèles acoustiques (AM) et des méthodes pour les tâches de traitement du signal ci-dessus. Cela nécessite la prise en compte du contexte musical et des informations musicales de haut niveau des LM en plus des caractéristiques acoustiques de bas niveau ci-dessus.\\
En outre, certaines expériences seront menées sur la base d'ensembles de données publiques, afin d'évaluer l'approche intégrée. Elles devraient couvrir certains cas de motifs rythmiques complexes se chevauchant.\\

Au-delà de l'intégration de modèles, il sera également intéressant d'étudier comment l'utilisation de LM peut améliorer les résultats de l'AM, voir [2], et ouvrir la voie à la génération entièrement automatisée de partitions de batterie et au problème général de l'AMT de bout en bout.
\cite{8350302}